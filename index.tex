% Options for packages loaded elsewhere
\PassOptionsToPackage{unicode}{hyperref}
\PassOptionsToPackage{hyphens}{url}
%
\documentclass[
  11pt,
]{book}
\usepackage[]{mathpazo}
\usepackage{amssymb,amsmath}
\usepackage{ifxetex,ifluatex}
\ifnum 0\ifxetex 1\fi\ifluatex 1\fi=0 % if pdftex
  \usepackage[T1]{fontenc}
  \usepackage[utf8]{inputenc}
  \usepackage{textcomp} % provide euro and other symbols
\else % if luatex or xetex
  \usepackage{unicode-math}
  \defaultfontfeatures{Scale=MatchLowercase}
  \defaultfontfeatures[\rmfamily]{Ligatures=TeX,Scale=1}
\fi
% Use upquote if available, for straight quotes in verbatim environments
\IfFileExists{upquote.sty}{\usepackage{upquote}}{}
\IfFileExists{microtype.sty}{% use microtype if available
  \usepackage[]{microtype}
  \UseMicrotypeSet[protrusion]{basicmath} % disable protrusion for tt fonts
}{}
\makeatletter
\@ifundefined{KOMAClassName}{% if non-KOMA class
  \IfFileExists{parskip.sty}{%
    \usepackage{parskip}
  }{% else
    \setlength{\parindent}{0pt}
    \setlength{\parskip}{6pt plus 2pt minus 1pt}}
}{% if KOMA class
  \KOMAoptions{parskip=half}}
\makeatother
\usepackage{xcolor}
\IfFileExists{xurl.sty}{\usepackage{xurl}}{} % add URL line breaks if available
\IfFileExists{bookmark.sty}{\usepackage{bookmark}}{\usepackage{hyperref}}
\hypersetup{
  pdftitle={Trabajo final Estadística Avanzada},
  pdfauthor={Carlos Alberto Murillo M; Luz Stella Florez; Diana Carolina Benjumea; Cindy Guerra},
  pdfkeywords={Maestria Ciencia Datos, ,},
  hidelinks,
  pdfcreator={LaTeX via pandoc}}
\urlstyle{same} % disable monospaced font for URLs
\usepackage[margin=1in]{geometry}
\usepackage{color}
\usepackage{fancyvrb}
\newcommand{\VerbBar}{|}
\newcommand{\VERB}{\Verb[commandchars=\\\{\}]}
\DefineVerbatimEnvironment{Highlighting}{Verbatim}{commandchars=\\\{\}}
% Add ',fontsize=\small' for more characters per line
\usepackage{framed}
\definecolor{shadecolor}{RGB}{248,248,248}
\newenvironment{Shaded}{\begin{snugshade}}{\end{snugshade}}
\newcommand{\AlertTok}[1]{\textcolor[rgb]{0.94,0.16,0.16}{#1}}
\newcommand{\AnnotationTok}[1]{\textcolor[rgb]{0.56,0.35,0.01}{\textbf{\textit{#1}}}}
\newcommand{\AttributeTok}[1]{\textcolor[rgb]{0.77,0.63,0.00}{#1}}
\newcommand{\BaseNTok}[1]{\textcolor[rgb]{0.00,0.00,0.81}{#1}}
\newcommand{\BuiltInTok}[1]{#1}
\newcommand{\CharTok}[1]{\textcolor[rgb]{0.31,0.60,0.02}{#1}}
\newcommand{\CommentTok}[1]{\textcolor[rgb]{0.56,0.35,0.01}{\textit{#1}}}
\newcommand{\CommentVarTok}[1]{\textcolor[rgb]{0.56,0.35,0.01}{\textbf{\textit{#1}}}}
\newcommand{\ConstantTok}[1]{\textcolor[rgb]{0.00,0.00,0.00}{#1}}
\newcommand{\ControlFlowTok}[1]{\textcolor[rgb]{0.13,0.29,0.53}{\textbf{#1}}}
\newcommand{\DataTypeTok}[1]{\textcolor[rgb]{0.13,0.29,0.53}{#1}}
\newcommand{\DecValTok}[1]{\textcolor[rgb]{0.00,0.00,0.81}{#1}}
\newcommand{\DocumentationTok}[1]{\textcolor[rgb]{0.56,0.35,0.01}{\textbf{\textit{#1}}}}
\newcommand{\ErrorTok}[1]{\textcolor[rgb]{0.64,0.00,0.00}{\textbf{#1}}}
\newcommand{\ExtensionTok}[1]{#1}
\newcommand{\FloatTok}[1]{\textcolor[rgb]{0.00,0.00,0.81}{#1}}
\newcommand{\FunctionTok}[1]{\textcolor[rgb]{0.00,0.00,0.00}{#1}}
\newcommand{\ImportTok}[1]{#1}
\newcommand{\InformationTok}[1]{\textcolor[rgb]{0.56,0.35,0.01}{\textbf{\textit{#1}}}}
\newcommand{\KeywordTok}[1]{\textcolor[rgb]{0.13,0.29,0.53}{\textbf{#1}}}
\newcommand{\NormalTok}[1]{#1}
\newcommand{\OperatorTok}[1]{\textcolor[rgb]{0.81,0.36,0.00}{\textbf{#1}}}
\newcommand{\OtherTok}[1]{\textcolor[rgb]{0.56,0.35,0.01}{#1}}
\newcommand{\PreprocessorTok}[1]{\textcolor[rgb]{0.56,0.35,0.01}{\textit{#1}}}
\newcommand{\RegionMarkerTok}[1]{#1}
\newcommand{\SpecialCharTok}[1]{\textcolor[rgb]{0.00,0.00,0.00}{#1}}
\newcommand{\SpecialStringTok}[1]{\textcolor[rgb]{0.31,0.60,0.02}{#1}}
\newcommand{\StringTok}[1]{\textcolor[rgb]{0.31,0.60,0.02}{#1}}
\newcommand{\VariableTok}[1]{\textcolor[rgb]{0.00,0.00,0.00}{#1}}
\newcommand{\VerbatimStringTok}[1]{\textcolor[rgb]{0.31,0.60,0.02}{#1}}
\newcommand{\WarningTok}[1]{\textcolor[rgb]{0.56,0.35,0.01}{\textbf{\textit{#1}}}}
\usepackage{graphicx,grffile}
\makeatletter
\def\maxwidth{\ifdim\Gin@nat@width>\linewidth\linewidth\else\Gin@nat@width\fi}
\def\maxheight{\ifdim\Gin@nat@height>\textheight\textheight\else\Gin@nat@height\fi}
\makeatother
% Scale images if necessary, so that they will not overflow the page
% margins by default, and it is still possible to overwrite the defaults
% using explicit options in \includegraphics[width, height, ...]{}
\setkeys{Gin}{width=\maxwidth,height=\maxheight,keepaspectratio}
% Set default figure placement to htbp
\makeatletter
\def\fps@figure{htbp}
\makeatother
\setlength{\emergencystretch}{3em} % prevent overfull lines
\providecommand{\tightlist}{%
  \setlength{\itemsep}{0pt}\setlength{\parskip}{0pt}}
\setcounter{secnumdepth}{-\maxdimen} % remove section numbering
\usepackage[]{natbib}
\bibliographystyle{apsr}

\title{Trabajo final Estadística Avanzada}
\author{Carlos Alberto Murillo M \and Luz Stella Florez \and Diana Carolina Benjumea \and Cindy Guerra}
\date{octubre 30, 2020}

\begin{document}
\frontmatter
\maketitle

{
\setcounter{tocdepth}{2}
\tableofcontents
}
\mainmatter
\hypertarget{abstract}{%
\chapter*{Abstract}\label{abstract}}
\addcontentsline{toc}{chapter}{Abstract}

Este documento anliza los impactos de las variables macoreconómicas en
los costos y gastos de una empresa en un determinado sector económico,
para nuestro estudio seleccionamos el sector minero, el código de este
trabajo se encuentra almacenado en el repositorio de Github :
\url{https://github.com/cabymetal/TrabajoFinal_estadisticos_avanzados}

\hypertarget{objetivos-y-lineamientos}{%
\chapter{Objetivos y Lineamientos}\label{objetivos-y-lineamientos}}

Caracterizar las relaciones entre algunos indicadores macroeconómicos y
los costos y gastos de ventas de las empresas colombianas vigiladas por
la SuperSociedades.

Lineamientos:

\begin{enumerate}
\def\labelenumi{\arabic{enumi}.}
\item
  Con ayuda de un modelo lineal modele cree un modelo o varios modelos
  que permitan caracterizar la relación entre las variables PIB,
  Inflación, Desempleo, Tasa de Cambio, Balance Fiscal, Balance en
  Cuenta Corriente, Tasa de intervención, TRM y los costos y gastos de
  ventas.
\item
  Se debe escoger mínimo un tipo de empresas (Clasificación Industrial
  Internacional Uniforme) que tenga más de 20 empresas y tomar al menos
  los últimos tres años de información disponible.
\item
  Se debe evaluar el ajuste y la capacidad predictiva.
\item
  Se deben explicar todas las transformaciones de variables requeridas
  por el modelo.
\item
  Se deben explicar todos los pasos para la construcción de la base de
  datos: descarga de información, concatenación, etc.
\item
  Se debe incluir un análisis descriptivo.
\item
  Se debe incluir un análsis de la razonabilidad de las cifras.
\item
  Se debe redactar un reporte técnico documentando lo anterior. La
  sugerencia es utilizar un formato que permita la inclusión de gráficos
  basados en html o JavaScript (por ejemplo hmtl a partir de Rmarkdown).
  El código se debe subir a un repositorio Git y referenciarlo en el
  reporte. El reporte debe incluir una estimación del esfuerzo de las
  actividades de 1) consolidación de información, 3) transformación de
  varibles y análisis descriptivo, 4) ajuste y validación de modelos y
  5) redacción del reporte.
\item
  El trabajo se debe subir al canal del curso en Teams y se debe
  notificar por correo a la dirección
  \href{mailto:judaospi@bancolombia.com.co}{\nolinkurl{judaospi@bancolombia.com.co}}.
\item
  La fecha de entrega es el viernes 30 de octubre y el trabajo se puede
  presentar en equipos de máximo cinco estudiantes.
\end{enumerate}

Para acceder a los datos de costos y gastos de ventas: • Entrar a
\url{http://pie.supersociedades.gov.co} \textgreater{} MENÚ
\textgreater{} Descarga Masiva de Información Descargar la información
de los años 2016 a 2019

\hypertarget{capuxedtulo-1.-lectura-de-variables-de-empresa}{%
\chapter{Capítulo 1. Lectura de variables de
empresa}\label{capuxedtulo-1.-lectura-de-variables-de-empresa}}

\begin{center}\rule{0.5\linewidth}{0.5pt}\end{center}

\hypertarget{selecciuxf3n-de-las-fuentes-de-informaciuxf3n}{%
\section{Selección de las fuentes de
información}\label{selecciuxf3n-de-las-fuentes-de-informaciuxf3n}}

Para los datos básicos y financieros de las empresas, tomamos los
siguientes archivos de la página de la Supersociedades:

\begin{itemize}
\item datosBasicosComplete.xlsx
\item Plenas - Individuales2017.xlsx
\item Plenas - Individuales2018.xlsx
\item Plenas - Individuales2019.xlsx
\end{itemize}

Primera iteración:

Código CIIU seleccionado: G4711

Macrosector: Comercio

Descripción: Comercio al por menor en establecimientos no especializados
con surtido compuesto principalmente por alimentos, bebidas o tabaco.

Esta clase incluye:

\begin{itemize}
\item Los establecimientos no especializados de comercio al por menor de productos cuyo surtido está compuesto principalmente de alimentos (víveres en general), bebidas o tabaco. No obstante, expenden otras mercancías para consumo de los hogares tales como vestuario, electrodomésticos, muebles, artículos de ferretería, cosméticos, entre otros. Suelen realizar este tipo de actividad los denominados supermercados, cooperativas de consumidores, comisariatos y otros establecimientos similares. También se incluyen las tiendas, los graneros, entre otros, que se encuentran en los pueblos o en barrios tradicionales.
\end{itemize}

Esta clase excluye:

\begin{itemize}
\item El expendio de comidas preparadas en restaurantes, cafeterías y por autoservicio.
\end{itemize}

Al realizar los cargues iniciales de información, nos dimos cuenta de
que cruzaban muy pocas empresas, el conjunto de datos seleccionado no
era suficiente, por lo que decidimos utilizar otro CIIU.

Segunda iteración:

Código CIIU seleccionado: B0722

Descripción: Extracción de oro y otros metales preciosos

Esta clase incluye:

\begin{itemize}
\item La extracción de oro, plata y otros metales del grupo del platino (osmio, iridio, rodio, rutenio y paladio).
\item Las actividades realizadas para extraer el oro existente en los lechos de los ríos sin importar el sistema de extracción empleado (barequeo, motobombas, draguetas, dragas, elevadores, monitores u otros).
\item La extracción de los metales preciosos se realiza a través de dos métodos: de veta o filón, que consiste en la extracción manual, mecanizada o semimecanizada de oro y de plata presentes en las rocas formando venas, vetas o filones.
\item Las actividades o procesos físicos necesarios para separar el oro de la roca que lo contiene, conocidos como procesos de beneficio del mineral, de los cuales los más comunes son la trituración y la molienda (pulverización).
\item Otros procesos tales como lavado (mazamorreo) hasta separar el oro y la plata de otros elementos o impurezas, siempre y cuando se realicen por cuenta del explotador y en sitios cercanos a la mina.
\item El segundo método consiste en la extracción de oro o platino de aluviones (concentración de mineral en el lecho de los ríos), el cual se realiza por diferentes sistemas de extracción, tales como: barequeo (mazamorreo); pequeña minería, representada por grupos de trabajadores que utilizan motobombas, elevadores y draguetas; mediana minería, utilizando maquinaria como retroexcavadoras y buldózeres, y la gran minería que realiza la extracción de metales preciosos por medio de dragas de cucharas.
\end{itemize}

Esta clase excluye:

\begin{itemize}
\item Los servicios de apoyo para la extracción de oro y metales preciosos. Se incluyen en la clase 0990, «Actividades de apoyo para otras actividades de explotación de minas y canteras».
\end{itemize}

\hypertarget{proceso-de-carga-de-los-datos}{%
\section{Proceso de carga de los
datos}\label{proceso-de-carga-de-los-datos}}

\begin{Shaded}
\begin{Highlighting}[]
\KeywordTok{library}\NormalTok{(tidyverse)}
\KeywordTok{library}\NormalTok{(}\StringTok{"readxl"}\NormalTok{)}
\KeywordTok{library}\NormalTok{(}\StringTok{"dplyr"}\NormalTok{)}
\end{Highlighting}
\end{Shaded}

\begin{enumerate}
\def\labelenumi{\arabic{enumi}.}
\tightlist
\item
  Cargamos los datos básicos de las empresas
\end{enumerate}

\begin{Shaded}
\begin{Highlighting}[]
\CommentTok{#Revisamos como son nuestros datos para saber si tenemos que realizar algún }
\CommentTok{#ajuste a la carga}
\CommentTok{#file.show("./data/datosBasicosComplete.xlsx")}

\CommentTok{#Como el archivo no tiene forma de tabla al principio, debemos realizar la }
\CommentTok{#carga, ignorando las primeras filas del archivo.}

\CommentTok{#Cargar los archivos a un dataframe}
\NormalTok{pd_datos_basicos <-}\StringTok{ }\KeywordTok{read_excel}\NormalTok{(}\StringTok{"./data/datosBasicosComplete.xlsx"}\NormalTok{, }
                               \DataTypeTok{sheet =} \StringTok{"Reporte"}\NormalTok{, }\DataTypeTok{skip=}\DecValTok{8}\NormalTok{, }\DataTypeTok{col_types =} \KeywordTok{c}\NormalTok{(}\StringTok{"text"}\NormalTok{, }
\StringTok{"text"}\NormalTok{, }\StringTok{"text"}\NormalTok{, }\StringTok{"text"}\NormalTok{, }\StringTok{"text"}\NormalTok{,}\StringTok{"text"}\NormalTok{,}\StringTok{"text"}\NormalTok{,}\StringTok{"text"}\NormalTok{,}\StringTok{"text"}\NormalTok{, }\StringTok{"text"}\NormalTok{,}\StringTok{"text"}\NormalTok{,}
\StringTok{"text"}\NormalTok{,}\StringTok{"text"}\NormalTok{,}\StringTok{"text"}\NormalTok{,}\StringTok{"text"}\NormalTok{,}\StringTok{"text"}\NormalTok{,}\StringTok{"date"}\NormalTok{,}\StringTok{"text"}\NormalTok{,}\StringTok{"date"}\NormalTok{,}\StringTok{"text"}\NormalTok{,}\StringTok{"date"}\NormalTok{, }\StringTok{"text"}\NormalTok{,}
\StringTok{"text"}\NormalTok{))}

\NormalTok{pd_datos_basicos }\OperatorTok\StringTok{ }
\StringTok{  }\KeywordTok{mutate}\NormalTok{(}\StringTok{`}\DataTypeTok{Órgano Societario}\StringTok{`}\NormalTok{ =}\StringTok{ }\KeywordTok{as.factor}\NormalTok{(}\StringTok{`}\DataTypeTok{Órgano Societario}\StringTok{`}\NormalTok{),}
      \StringTok{`}\DataTypeTok{Etapa Situación` = as.factor(}\StringTok{`}\NormalTok{Etapa Situación`)}\ErrorTok{)}\NormalTok{ ->}\StringTok{ }\NormalTok{pd_datos_basicos}

\KeywordTok{head}\NormalTok{(pd_datos_basicos)}
\end{Highlighting}
\end{Shaded}

\begin{verbatim}
## # A tibble: 6 x 23
##   NIT   `Razón social` `Código CIIU` `Tipo Societari~ `Objeto Social`
##   <chr> <chr>          <chr>         <chr>            <chr>          
## 1 1001~ NOREÑA  MANRI~ 0             PERSONA NATURAL  <NA>           
## 2 1001~ PEÑA RAMIREZ ~ H5229         PERSONA NATURAL  <NA>           
## 3 1002~ GONZALEZ SANC~ G4731         PERSONA NATURAL  <NA>           
## 4 1002~ RODRIGO JAVIE~ L6810         PERSONA NATURAL  <NA>           
## 5 1002~ BUITRAGO GONZ~ H4923         PERSONA NATURAL  <NA>           
## 6 1005~ KAREN JULIETH~ M7500         PERSONA NATURAL  <NA>           
## # ... with 18 more variables: `Dirección Notificación Judicial` <chr>, `Ciudad
## #   Notificación Judicial` <chr>, `Departamento Notificación Judicial` <chr>,
## #   `Teléfono Notificación Judicial` <chr>, `Dirección Domicilio` <chr>,
## #   `Ciudad Domicilio` <chr>, `Departamento Domicilio` <chr>, `Apartado
## #   Domicilio` <chr>, `E-Mail` <chr>, Web <chr>, Estado <chr>, `Fecha
## #   Estado` <dttm>, Situación <chr>, `Fecha Situación` <dttm>, `Etapa
## #   Situación` <fct>, `Fecha Etapa` <dttm>, `Nombre Representante Legal` <chr>,
## #   `Órgano Societario` <fct>
\end{verbatim}

\begin{enumerate}
\def\labelenumi{\arabic{enumi}.}
\setcounter{enumi}{1}
\tightlist
\item
  Filtramos los datos del CIIU seleccionado
\end{enumerate}

\begin{Shaded}
\begin{Highlighting}[]
\KeywordTok{library}\NormalTok{(dplyr)}

\NormalTok{pd_datos_basicos_flt <-}\StringTok{ }\NormalTok{pd_datos_basicos[,}\KeywordTok{c}\NormalTok{(}\StringTok{"NIT"}\NormalTok{,}\StringTok{"Razón social"}\NormalTok{,}\StringTok{"Código CIIU"}\NormalTok{,}
      \StringTok{"Ciudad Domicilio"}\NormalTok{,}\StringTok{"Departamento Domicilio"}\NormalTok{, }\StringTok{"Estado"}\NormalTok{,}\StringTok{"Situación", }
\StringTok{      "}\NormalTok{Órgano Societario}\StringTok{", "}\NormalTok{Etapa Situación")]}

\KeywordTok{names}\NormalTok{ (pd_datos_basicos_flt) =}\StringTok{ }\KeywordTok{c}\NormalTok{(}\StringTok{"NIT"}\NormalTok{,}\StringTok{"razon_social"}\NormalTok{,}\StringTok{"CIIU"}\NormalTok{,}\StringTok{"ciudad"}\NormalTok{,}
                                 \StringTok{"departamento"}\NormalTok{, }\StringTok{"estado"}\NormalTok{,}\StringTok{"situacion"}\NormalTok{, }
                                 \StringTok{"organo_societario"}\NormalTok{, }\StringTok{"etapa_situacion"}\NormalTok{)}

\NormalTok{pd_datos_basicos_flt <-}\StringTok{ }\KeywordTok{filter}\NormalTok{(pd_datos_basicos_flt, CIIU }\OperatorTok{==}\StringTok{ "B0722"} \OperatorTok{&}\StringTok{ }
\StringTok{                                 }\NormalTok{situacion }\OperatorTok{==}\StringTok{ "ACTIVA"}\NormalTok{)}

\KeywordTok{head}\NormalTok{(pd_datos_basicos_flt)}
\end{Highlighting}
\end{Shaded}

\begin{verbatim}
## # A tibble: 6 x 9
##   NIT   razon_social CIIU  ciudad departamento estado situacion organo_societar~
##   <chr> <chr>        <chr> <chr>  <chr>        <chr>  <chr>     <fct>           
## 1 8002~ GRUPO DE BU~ B0722 MEDEL~ ANTIOQUIA    INSPE~ ACTIVA    ACTIVIDAD ECONO~
## 2 8110~ MINERA CROE~ B0722 MEDEL~ ANTIOQUIA    INSPE~ ACTIVA    ACTIVIDAD ECONO~
## 3 8110~ NUEVA CALIF~ B0722 MEDEL~ ANTIOQUIA    INSPE~ ACTIVA    ACTIVIDAD ECONO~
## 4 8110~ COLOMBIA GO~ B0722 MEDEL~ ANTIOQUIA    INSPE~ ACTIVA    ACTIVIDAD ECONO~
## 5 8110~ NEGOCIOS MI~ B0722 MEDEL~ ANTIOQUIA    INSPE~ ACTIVA    ACTIVIDAD ECONO~
## 6 8300~ ECO ORO MIN~ B0722 BUCAR~ SANTANDER    INSPE~ ACTIVA    ACTIVIDAD ECONO~
## # ... with 1 more variable: etapa_situacion <fct>
\end{verbatim}

\begin{enumerate}
\def\labelenumi{\arabic{enumi}.}
\setcounter{enumi}{2}
\tightlist
\item
  Cargamos los datos financieros
\end{enumerate}

\begin{Shaded}
\begin{Highlighting}[]
\NormalTok{pd_datos_fin_}\DecValTok{2017}\NormalTok{ <-}\StringTok{ }\KeywordTok{read_excel}\NormalTok{(}\StringTok{"./data/Plenas - Individuales2017.xlsx"}\NormalTok{, }
                                \DataTypeTok{sheet =} \StringTok{"Estado de Resultado Integral"}\NormalTok{ )}

\NormalTok{pd_datos_fin_}\DecValTok{2017}\NormalTok{ <-}\StringTok{ }\NormalTok{pd_datos_fin_}\DecValTok{2017}\NormalTok{[,}\KeywordTok{c}\NormalTok{(}\StringTok{"Nit"}\NormalTok{, }\StringTok{"Periodo"}\NormalTok{, }\StringTok{"Costo de ventas"}\NormalTok{,  }
\StringTok{"Costos de distribución", "}\NormalTok{Gastos de administración", }\StringTok{"Otros gastos, por función",}
\StringTok{"}\NormalTok{Costos financieros}\StringTok{", "}\KeywordTok{Gasto}\NormalTok{ (ingreso) por impuestos, operaciones continuadas}\StringTok{",}
\StringTok{"}\NormalTok{Ingresos de actividades ordinarias}\StringTok{", "}\NormalTok{Otros ingresos}\StringTok{", "}\NormalTok{Ingresos financieros}\StringTok{")]}


\StringTok{names (pd_datos_fin_2017) = c("}\NormalTok{NIT}\StringTok{", "}\NormalTok{Periodo}\StringTok{", "}\NormalTok{costo_ventas}\StringTok{",  }
\StringTok{  "}\NormalTok{costo_distribucion}\StringTok{", "}\NormalTok{gastos_administracion}\StringTok{", "}\NormalTok{otros_gastos}\StringTok{", }
\StringTok{  "}\NormalTok{costos_financieros}\StringTok{", "}\NormalTok{gasto_impuestos_operaciones}\StringTok{", }
\StringTok{  "}\NormalTok{ingresos_actividades_ordinarias}\StringTok{", "}\NormalTok{otros_ingresos}\StringTok{", "}\NormalTok{ingresos_financieros}\StringTok{")}


\StringTok{datos_completos_fin <- merge (pd_datos_basicos_flt, pd_datos_fin_2017, }
\StringTok{                              by.x="}\NormalTok{NIT}\StringTok{", by.y="}\NormalTok{NIT}\StringTok{")}
\end{Highlighting}
\end{Shaded}

Para efectos del ejercicio, no tomaremos el archivo de 2017, ya que el
archivo 2018 tiene los datos de 2017 con la nueva norma.

\begin{Shaded}
\begin{Highlighting}[]
\NormalTok{pd_datos_fin_}\DecValTok{2018}\NormalTok{ <-}\StringTok{ }\KeywordTok{read_excel}\NormalTok{(}\StringTok{"./data/Plenas - Individuales2018.xlsx"}\NormalTok{, }
                                \DataTypeTok{sheet =} \StringTok{"ERI"}\NormalTok{ )}

\NormalTok{pd_datos_fin_}\DecValTok{2018}\NormalTok{ <-}\StringTok{ }\NormalTok{pd_datos_fin_}\DecValTok{2018}\NormalTok{[,}\KeywordTok{c}\NormalTok{(}\StringTok{"Nit"}\NormalTok{, }\StringTok{"Periodo"}\NormalTok{, }\StringTok{"Costo de ventas"}\NormalTok{, }
  \StringTok{"Gastos de ventas"}\NormalTok{, }\StringTok{"Gastos de administración", "}\NormalTok{Otros gastos}\StringTok{", }
\StringTok{  "}\NormalTok{Costos financieros}\StringTok{", "}\KeywordTok{Ingreso}\NormalTok{ (gasto) por impuestos}\StringTok{", }
\StringTok{  "}\NormalTok{Ingresos de actividades ordinarias}\StringTok{", "}\NormalTok{Otros ingresos}\StringTok{", "}\NormalTok{Ingresos financieros}\StringTok{")]}

\StringTok{names (pd_datos_fin_2018) = c("}\NormalTok{NIT}\StringTok{", "}\NormalTok{Periodo}\StringTok{", "}\NormalTok{costo_ventas}\StringTok{", "}\NormalTok{gastos_ventas}\StringTok{",}
\StringTok{  "}\NormalTok{gastos_administracion}\StringTok{", "}\NormalTok{otros_gastos}\StringTok{", "}\NormalTok{costos_financieros}\StringTok{", "}\NormalTok{gasto_impuestos}\StringTok{",}
\StringTok{  "}\NormalTok{ingresos_actividades_ordinarias}\StringTok{", "}\NormalTok{otros_ingresos}\StringTok{", "}\NormalTok{ingresos_financieros}\StringTok{" )}

\StringTok{datos_completos_2018 <- merge (pd_datos_basicos_flt, pd_datos_fin_2018,}
\StringTok{                               by.x="}\NormalTok{NIT}\StringTok{", by.y="}\NormalTok{NIT}\StringTok{")}

\StringTok{#Le damos formato a los periodos}

\StringTok{datos_completos_2018$Periodo[datos_completos_2018$Periodo == "}\NormalTok{Periodo Anterior}\StringTok{"] <- }
\StringTok{  "}\DecValTok{2017}\StringTok{"}
\StringTok{datos_completos_2018$Periodo[datos_completos_2018$Periodo == "}\NormalTok{Periodo Actual}\StringTok{"] <- }
\StringTok{  "}\DecValTok{2018}\StringTok{"}
\end{Highlighting}
\end{Shaded}

\begin{Shaded}
\begin{Highlighting}[]
\NormalTok{pd_datos_fin_}\DecValTok{2019}\NormalTok{ <-}\StringTok{ }\KeywordTok{read_excel}\NormalTok{(}\StringTok{"./data/Plenas - Individuales2019.xlsx"}\NormalTok{,}
                                \DataTypeTok{sheet =} \StringTok{"ERI"}\NormalTok{ )}

\CommentTok{#Revisar Costos de distribución}
\NormalTok{pd_datos_fin_}\DecValTok{2019}\NormalTok{ <-}\StringTok{ }\NormalTok{pd_datos_fin_}\DecValTok{2019}\NormalTok{[,}\KeywordTok{c}\NormalTok{(}\StringTok{"Nit"}\NormalTok{, }\StringTok{"Periodo"}\NormalTok{, }\StringTok{"Costo de ventas"}\NormalTok{, }
  \StringTok{"Gastos de administración", "}\NormalTok{Otros gastos}\StringTok{", "}\NormalTok{Costos financieros}\StringTok{", }
\StringTok{  "}\KeywordTok{Ingreso}\NormalTok{ (gasto) por impuestos}\StringTok{", "}\NormalTok{Ingresos de actividades ordinarias}\StringTok{", }
\StringTok{  "}\NormalTok{Otros ingresos}\StringTok{", "}\NormalTok{Ingresos financieros}\StringTok{")]}

\StringTok{names (pd_datos_fin_2019) = c("}\NormalTok{NIT}\StringTok{", "}\NormalTok{Periodo}\StringTok{", "}\NormalTok{costo_ventas}\StringTok{", }
\StringTok{  "}\NormalTok{gastos_administracion}\StringTok{", "}\NormalTok{otros_gastos}\StringTok{", "}\NormalTok{costos_financieros}\StringTok{", }
\StringTok{  "}\NormalTok{gasto_impuestos}\StringTok{", "}\NormalTok{ingresos_actividades_ordinarias}\StringTok{", "}\NormalTok{otros_ingresos}\StringTok{",}
\StringTok{  "}\NormalTok{ingresos_financieros}\StringTok{" )}

\StringTok{datos_completos <- merge (pd_datos_basicos_flt, pd_datos_fin_2019,}
\StringTok{                          by.x="}\NormalTok{NIT}\StringTok{", by.y="}\NormalTok{NIT}\StringTok{")}

\StringTok{datos_completos$Periodo[datos_completos$Periodo == "}\NormalTok{Periodo Actual}\StringTok{"] <- "}\DecValTok{2019}\StringTok{"}

\StringTok{datos_completos <- filter(datos_completos, Periodo == "}\DecValTok{2019}\StringTok{")}
\end{Highlighting}
\end{Shaded}

\begin{Shaded}
\begin{Highlighting}[]
\CommentTok{#Eliminamos variable diferente a 2019}
\NormalTok{datos_completos_}\DecValTok{2018}\NormalTok{ <-}\StringTok{ }\KeywordTok{select}\NormalTok{(datos_completos_}\DecValTok{2018}\NormalTok{, }\OperatorTok{-}\NormalTok{gastos_ventas)}

\CommentTok{#UNimos los 2 dataframes}
\NormalTok{datos_completos =}\StringTok{ }\KeywordTok{rbind}\NormalTok{(datos_completos, datos_completos_}\DecValTok{2018}\NormalTok{)}
\NormalTok{datos_completos }\OperatorTok\StringTok{ }\KeywordTok{mutate}\NormalTok{( }\DataTypeTok{razon_social =} \KeywordTok{as.factor}\NormalTok{(razon_social), }
                            \DataTypeTok{CIIU =} \KeywordTok{as.factor}\NormalTok{(CIIU), }\DataTypeTok{ciudad =} \KeywordTok{as.factor}\NormalTok{(ciudad),}
                          \DataTypeTok{departamento =} \KeywordTok{as.factor}\NormalTok{(departamento), }
                          \DataTypeTok{estado =}\KeywordTok{as.factor}\NormalTok{(estado), }\DataTypeTok{Periodo =} \KeywordTok{as.factor}\NormalTok{(Periodo),}
                          \DataTypeTok{situacion =} \KeywordTok{as.factor}\NormalTok{(situacion)) ->}\StringTok{ }\NormalTok{datos_completos}
\end{Highlighting}
\end{Shaded}

\hypertarget{capuxedtulo-2.-lectura-y-consolidaciuxf3n-de-variables-econuxf3micas}{%
\chapter{Capítulo 2. Lectura y consolidación de variables
económicas}\label{capuxedtulo-2.-lectura-y-consolidaciuxf3n-de-variables-econuxf3micas}}

\textbf{A continuación se preseenta el proceso que se ejecutó para generar un dataframe con las variables de PIB, Inflación, Desempleo, Balance Fiscal, Balance en Cuenta Corriente, Tasa de intervención, TRM}

\textbf{Para el PIB:} Es un indicador económico que refleja el valor
monetario de todos los bienes y servicios finales producidos por un país
o región en un determinado periodo de tiempo, normalmente un año. Se
utiliza para medir la riqueza que genera un país.

\begin{Shaded}
\begin{Highlighting}[]
\KeywordTok{library}\NormalTok{(dplyr)}
\CommentTok{#Los datos son tomados de https://datosmacro.expansion.com/pib/colombia}
\CommentTok{# vectores }
\NormalTok{anyo <-}\StringTok{ }\KeywordTok{c}\NormalTok{(}\StringTok{"2016"}\NormalTok{, }\StringTok{"2017"}\NormalTok{, }\StringTok{"2018"}\NormalTok{, }\StringTok{"2019"}\NormalTok{)}
\NormalTok{PIB_M.E. <-}\StringTok{ }\KeywordTok{c}\NormalTok{(}\FloatTok{289.239}\NormalTok{, }\FloatTok{280.249}\NormalTok{, }\FloatTok{275.999}\NormalTok{, }\FloatTok{255.416}\NormalTok{)}
\NormalTok{Var.PIB <-}\StringTok{ }\KeywordTok{c}\NormalTok{(}\FloatTok{3.3}\NormalTok{, }\FloatTok{2.5}\NormalTok{, }\FloatTok{1.4}\NormalTok{, }\FloatTok{2.1}\NormalTok{)}
\CommentTok{#Crear dataframe de vectores}
\NormalTok{PIB <-}\StringTok{ }\KeywordTok{data.frame}\NormalTok{(anyo, PIB_M.E., Var.PIB)}
\KeywordTok{head}\NormalTok{(PIB)}
\end{Highlighting}
\end{Shaded}

\begin{verbatim}
##   anyo PIB_M.E. Var.PIB
## 1 2016  289.239     3.3
## 2 2017  280.249     2.5
## 3 2018  275.999     1.4
## 4 2019  255.416     2.1
\end{verbatim}

\textbf{Para la inflación:} La inflación es un fenómeno que se observa
en la economía de un país y está relacionado con el aumento desordenado
de los precios de la mayor parte de los bienes y servicios que se
comercian en sus mercados, por un periodo de tiempo prolongado.
\href{https://es.wikipedia.org/wiki/Anexo:Variaci\%C3\%B3n_de_la_inflaci\%C3\%B3n_de_Colombia_desde_1946}{datos
tomados de aquí}

\begin{Shaded}
\begin{Highlighting}[]
\CommentTok{# vectores }
\NormalTok{anyo <-}\StringTok{ }\KeywordTok{c}\NormalTok{(}\StringTok{"2016"}\NormalTok{, }\StringTok{"2017"}\NormalTok{, }\StringTok{"2018"}\NormalTok{, }\StringTok{"2019"}\NormalTok{)}
\NormalTok{Inflacion <-}\StringTok{ }\KeywordTok{c}\NormalTok{(}\FloatTok{5.75}\NormalTok{, }\FloatTok{4.09}\NormalTok{, }\FloatTok{3.18}\NormalTok{, }\FloatTok{3.80}\NormalTok{)}
\CommentTok{#Crear dataframe de vectores}
\NormalTok{Inflacion <-}\StringTok{ }\KeywordTok{data.frame}\NormalTok{(anyo, Inflacion)}
\KeywordTok{head}\NormalTok{(Inflacion)}
\end{Highlighting}
\end{Shaded}

\begin{verbatim}
##   anyo Inflacion
## 1 2016      5.75
## 2 2017      4.09
## 3 2018      3.18
## 4 2019      3.80
\end{verbatim}

\textbf{Para el desempleo:} Es otra de las variables mas importantes de
la macroeconomía, porque afecta directamente el bienestar de las
personas. El desempleo es el porcentaje de la fuerza de trabajo que está
buscando trabajo activamente y que actualmente se encuentra desempleada.
\href{https://www.dane.gov.co/index.php/estadisticas-por-tema/mercado-laboral/empleo-y-desempleo}{datos
tomados de aquí}

\begin{Shaded}
\begin{Highlighting}[]
\CommentTok{# vectores }
\NormalTok{anyo <-}\StringTok{ }\KeywordTok{c}\NormalTok{(}\StringTok{"2016"}\NormalTok{, }\StringTok{"2017"}\NormalTok{, }\StringTok{"2018"}\NormalTok{, }\StringTok{"2019"}\NormalTok{)}
\NormalTok{Desempleo <-}\StringTok{ }\KeywordTok{c}\NormalTok{(}\FloatTok{9.2}\NormalTok{, }\FloatTok{9.4}\NormalTok{, }\FloatTok{9.7}\NormalTok{, }\FloatTok{10.5}\NormalTok{)}
\NormalTok{Var.Desempleo <-}\StringTok{ }\KeywordTok{c}\NormalTok{(}\FloatTok{3.36}\NormalTok{, }\FloatTok{1.99}\NormalTok{, }\FloatTok{3.19}\NormalTok{, }\FloatTok{8.25}\NormalTok{)}
\CommentTok{#Crear dataframe de vectores}
\NormalTok{Desempleo <-}\StringTok{ }\KeywordTok{data.frame}\NormalTok{(anyo, Desempleo, Var.Desempleo)}
\KeywordTok{head}\NormalTok{(Desempleo)}
\end{Highlighting}
\end{Shaded}

\begin{verbatim}
##   anyo Desempleo Var.Desempleo
## 1 2016       9.2          3.36
## 2 2017       9.4          1.99
## 3 2018       9.7          3.19
## 4 2019      10.5          8.25
\end{verbatim}

\textbf{Para el balance fiscal:} Es la diferencia entre ingresos y
gastos públicos en un determinado territorio.
\href{http://www.urf.gov.co/webcenter/portal/EntidadesFinancieras/pages_EntidadesFinancieras/PoliticaFiscal/bgg/balancefiscalgobiernocentral?_afrLoop=6729623401772216\&_afrWindowMode=2\&Adf-Window-Id=mof7t7k7j\&_afrFS=16\&_afrMT=screen\&_afrMFW=768\&_afrMFH=720\&_afrMFDW=1536\&_afrMFDH=864\&_afrMFC=8\&_afrMFCI=0\&_afrMFM=0\&_afrMFR=120\&_afrMFG=0\&_afrMFS=0\&_afrMFO=0}{datos
tomados de aquí}

\begin{Shaded}
\begin{Highlighting}[]
\CommentTok{# vectores }
\NormalTok{anyo <-}\StringTok{ }\KeywordTok{c}\NormalTok{(}\StringTok{"2016"}\NormalTok{, }\StringTok{"2017"}\NormalTok{, }\StringTok{"2018"}\NormalTok{, }\StringTok{"2019"}\NormalTok{)}
\NormalTok{GNC <-}\StringTok{ }\KeywordTok{c}\NormalTok{(}\OperatorTok{-}\DecValTok{4}\NormalTok{, }\FloatTok{-3.6}\NormalTok{, }\FloatTok{-3.1}\NormalTok{, }\FloatTok{-2.5}\NormalTok{)}
\CommentTok{#Crear dataframe de vectores}
\NormalTok{GNC <-}\StringTok{ }\KeywordTok{data.frame}\NormalTok{(anyo,GNC)}
\KeywordTok{head}\NormalTok{(GNC)}
\end{Highlighting}
\end{Shaded}

\begin{verbatim}
##   anyo  GNC
## 1 2016 -4.0
## 2 2017 -3.6
## 3 2018 -3.1
## 4 2019 -2.5
\end{verbatim}

\textbf{Para el balance en cuenta corriente:} Es el conjunto de
transacciones de intercambio de bienes y servicios, rentas y
transferencias (tanto corrientes como de capital), su saldo determina la
capacidad o necesidad de financiación de un país.

\begin{Shaded}
\begin{Highlighting}[]
\CommentTok{# vectores }
\NormalTok{anyo <-}\StringTok{ }\KeywordTok{c}\NormalTok{(}\StringTok{"2016"}\NormalTok{, }\StringTok{"2017"}\NormalTok{, }\StringTok{"2018"}\NormalTok{, }\StringTok{"2019"}\NormalTok{)}
\NormalTok{Balance_Cuenta_Corriente <-}\StringTok{ }\KeywordTok{c}\NormalTok{(}\OperatorTok{-}\FloatTok{13747.75}\NormalTok{, }\FloatTok{-13117.66}\NormalTok{, }\FloatTok{-10240.88}\NormalTok{, }\FloatTok{-12036.18}\NormalTok{)}
\CommentTok{#Crear dataframe de vectores}
\NormalTok{Balance_Cuenta_Corriente <-}\StringTok{ }\KeywordTok{data.frame}\NormalTok{(anyo, Balance_Cuenta_Corriente)}
\KeywordTok{head}\NormalTok{(Balance_Cuenta_Corriente)}
\end{Highlighting}
\end{Shaded}

\begin{verbatim}
##   anyo Balance_Cuenta_Corriente
## 1 2016                -13747.75
## 2 2017                -13117.66
## 3 2018                -10240.88
## 4 2019                -12036.18
\end{verbatim}

\textbf{Para la tasa de intervención:} Corresponde a la tasa de interés
mínima que le cobra el Banco de la República a las entidades financieras
por los préstamos que les concede generalmente a un día y, además, sirve
como referencia para establecer la tasa de interés máxima que les paga
por recibirles dinero que tengan como excedente.
\href{https://www.banrep.gov.co/es/estadisticas/tasas-interes-politica-monetaria}{datos
tomados de aquí}

\begin{Shaded}
\begin{Highlighting}[]
\CommentTok{# vectores }
\NormalTok{anyo <-}\StringTok{ }\KeywordTok{c}\NormalTok{(}\StringTok{"2016"}\NormalTok{, }\StringTok{"2017"}\NormalTok{, }\StringTok{"2018"}\NormalTok{, }\StringTok{"2019"}\NormalTok{)}
\NormalTok{TIM_promedio<-}\StringTok{ }\KeywordTok{c}\NormalTok{(}\FloatTok{7.10}\NormalTok{, }\FloatTok{6.13}\NormalTok{, }\FloatTok{4.35}\NormalTok{, }\FloatTok{4.25}\NormalTok{)}
\CommentTok{#Crear dataframe de vectores}
\NormalTok{TIM <-}\StringTok{ }\KeywordTok{data.frame}\NormalTok{(anyo, TIM_promedio)}
\KeywordTok{head}\NormalTok{(TIM)}
\end{Highlighting}
\end{Shaded}

\begin{verbatim}
##   anyo TIM_promedio
## 1 2016         7.10
## 2 2017         6.13
## 3 2018         4.35
## 4 2019         4.25
\end{verbatim}

\textbf{Para la TRM:} La tasa de cambio representativa del mercado (TRM)
es la cantidad de pesos colombianos por un dólar de los Estados Unidos.
La TRM se calcula con base en las operaciones de compra y venta de
divisas entre intermediarios financieros que transan en el mercado
cambiario colombiano, con cumplimiento el mismo día cuando se realiza la
negociación de las divisas.

Actualmente la Superintendencia Financiera de Colombia es la que calcula
y certifica diariamente la TRM con base en las operaciones registradas
el día hábil inmediatamente anterior.
\href{https://www.dolar-colombia.com/historico}{datos tomados de aquí}

\begin{Shaded}
\begin{Highlighting}[]
\CommentTok{#Se leen los datos -  }
\NormalTok{dataset =}\StringTok{ }\KeywordTok{read.csv}\NormalTok{(}\StringTok{'./data/TRM.csv'}\NormalTok{, }\DataTypeTok{check.names =} \OtherTok{FALSE}\NormalTok{, }\DataTypeTok{encoding =} \StringTok{"UTF-8"}\NormalTok{, }
                   \DataTypeTok{blank.lines.skip =} \OtherTok{FALSE}\NormalTok{, }\DataTypeTok{dec=}\StringTok{","}\NormalTok{)}

\CommentTok{#se conservam unicamente las columnas de año y TRM}
\NormalTok{df =}\StringTok{ }\NormalTok{dataset[}\DecValTok{1}\NormalTok{]}
\NormalTok{df[}\StringTok{'TRM'}\NormalTok{] =}\StringTok{ }\NormalTok{dataset[}\DecValTok{3}\NormalTok{]}
\NormalTok{df}\OperatorTok{$}\NormalTok{TRM <-}\StringTok{ }\KeywordTok{as.numeric}\NormalTok{(}\KeywordTok{as.character}\NormalTok{(df}\OperatorTok{$}\NormalTok{TRM))}

\CommentTok{#Se agrupa bajo la media}
\NormalTok{media =}\StringTok{ }\NormalTok{df}
\NormalTok{media =}\StringTok{ }\NormalTok{media }\OperatorTok
\StringTok{  }\KeywordTok{group_by}\NormalTok{(media[}\DecValTok{1}\NormalTok{]) }\OperatorTok
\StringTok{  }\KeywordTok{summarise}\NormalTok{(}\KeywordTok{across}\NormalTok{(}\DataTypeTok{.cols =} \KeywordTok{everything}\NormalTok{(), }\DataTypeTok{.fns =}\NormalTok{ mean))}
\end{Highlighting}
\end{Shaded}

\begin{verbatim}
## `summarise()` ungrouping output (override with `.groups` argument)
\end{verbatim}

\begin{Shaded}
\begin{Highlighting}[]
\CommentTok{#para la mediana}
\NormalTok{mediana =}\StringTok{ }\NormalTok{df}
\NormalTok{mediana =}\StringTok{ }\NormalTok{mediana }\OperatorTok
\StringTok{  }\KeywordTok{group_by}\NormalTok{(mediana[}\DecValTok{1}\NormalTok{]) }\OperatorTok
\StringTok{  }\KeywordTok{summarise}\NormalTok{(}\KeywordTok{across}\NormalTok{(}\DataTypeTok{.cols =} \KeywordTok{everything}\NormalTok{(), }\DataTypeTok{.fns =}\NormalTok{ median))}
\end{Highlighting}
\end{Shaded}

\begin{verbatim}
## `summarise()` ungrouping output (override with `.groups` argument)
\end{verbatim}

\begin{Shaded}
\begin{Highlighting}[]
\CommentTok{#Se genera un dataframe con los datos obtenidos}
\NormalTok{df =}\StringTok{ }\NormalTok{media}
\KeywordTok{colnames}\NormalTok{(df)[}\DecValTok{2}\NormalTok{] <-}\StringTok{ 'TRM_media'}
\NormalTok{df[}\StringTok{'TRM_mediana'}\NormalTok{] <-}\StringTok{ }\NormalTok{mediana[}\DecValTok{2}\NormalTok{]}
\NormalTok{df}
\end{Highlighting}
\end{Shaded}

\begin{verbatim}
## # A tibble: 4 x 3
##    Anyo TRM_media TRM_mediana
##   <int>     <dbl>       <dbl>
## 1  2016     3051.       3003.
## 2  2017     2951.       2942.
## 3  2018     2956.       2898.
## 4  2019     3281.       3277.
\end{verbatim}

Se unen los datos en un solo dataframe

\begin{Shaded}
\begin{Highlighting}[]
\NormalTok{df[}\StringTok{'PIB_M.E.'}\NormalTok{] =}\StringTok{ }\NormalTok{PIB[}\DecValTok{2}\NormalTok{]}
\NormalTok{df[}\StringTok{'Var.PIB'}\NormalTok{] =}\StringTok{ }\NormalTok{PIB[}\DecValTok{3}\NormalTok{]}
\NormalTok{df[}\StringTok{'Inflacion'}\NormalTok{] =}\StringTok{ }\NormalTok{Inflacion[}\DecValTok{2}\NormalTok{]}
\NormalTok{df[}\StringTok{'Desempleo'}\NormalTok{] =}\StringTok{ }\NormalTok{Desempleo[}\DecValTok{2}\NormalTok{]}
\NormalTok{df[}\StringTok{'var.Desempleo'}\NormalTok{] =}\StringTok{ }\NormalTok{Desempleo[}\DecValTok{3}\NormalTok{]}
\NormalTok{df[}\StringTok{'GNC'}\NormalTok{] =}\StringTok{ }\NormalTok{GNC[}\DecValTok{2}\NormalTok{]}
\NormalTok{df[}\StringTok{'Balance_Cuenta_Corriente'}\NormalTok{] =}\StringTok{ }\NormalTok{Balance_Cuenta_Corriente[}\DecValTok{2}\NormalTok{]}
\NormalTok{df[}\StringTok{'TIM_promedio'}\NormalTok{] =}\StringTok{ }\NormalTok{TIM[}\DecValTok{2}\NormalTok{]}
\KeywordTok{head}\NormalTok{(df)}
\end{Highlighting}
\end{Shaded}

\begin{verbatim}
## # A tibble: 4 x 11
##    Anyo TRM_media TRM_mediana PIB_M.E. Var.PIB Inflacion Desempleo var.Desempleo
##   <int>     <dbl>       <dbl>    <dbl>   <dbl>     <dbl>     <dbl>         <dbl>
## 1  2016     3051.       3003.     289.     3.3      5.75       9.2          3.36
## 2  2017     2951.       2942.     280.     2.5      4.09       9.4          1.99
## 3  2018     2956.       2898.     276.     1.4      3.18       9.7          3.19
## 4  2019     3281.       3277.     255.     2.1      3.8       10.5          8.25
## # ... with 3 more variables: GNC <dbl>, Balance_Cuenta_Corriente <dbl>,
## #   TIM_promedio <dbl>
\end{verbatim}

\hypertarget{capuxedtulo-3.-consolidaciuxf3n-de-la-base}{%
\chapter{Capítulo 3. Consolidación de la
base}\label{capuxedtulo-3.-consolidaciuxf3n-de-la-base}}

En esta sección se unen las dos bases generadas en las fases anteriores
en una sola base

\begin{Shaded}
\begin{Highlighting}[]
\KeywordTok{library}\NormalTok{(dplyr)}
\KeywordTok{library}\NormalTok{(tidyr)}
\NormalTok{datos_completos}\OperatorTok\KeywordTok{mutate}\NormalTok{(}\DataTypeTok{Periodo=}\KeywordTok{as.numeric}\NormalTok{(}\KeywordTok{as.character}\NormalTok{(Periodo))) }\OperatorTok\StringTok{ }
\StringTok{  }\KeywordTok{inner_join}\NormalTok{(df, }\DataTypeTok{by=}\KeywordTok{c}\NormalTok{(}\StringTok{"Periodo"}\NormalTok{ =}\StringTok{ "Anyo"}\NormalTok{)) }\OperatorTok\StringTok{ }\KeywordTok{replace_na}\NormalTok{(}\KeywordTok{list}\NormalTok{(}\DataTypeTok{costo_ventas =} \DecValTok{0}\NormalTok{, }
  \DataTypeTok{gastos_administracion =} \DecValTok{0}\NormalTok{, }\DataTypeTok{otros_gastos=}\DecValTok{0}\NormalTok{, }\DataTypeTok{costos_financieros=}\DecValTok{0}\NormalTok{ , }
  \DataTypeTok{gasto_impuestos=}\DecValTok{0}\NormalTok{, }\DataTypeTok{ingresos_actividades_ordinarias=}\DecValTok{0}\NormalTok{, }\DataTypeTok{otros_ingresos =} \DecValTok{0}\NormalTok{, }
  \DataTypeTok{ingresos_financieros=}\DecValTok{0}\NormalTok{)) ->}\StringTok{ }\NormalTok{datos_completos2}


\NormalTok{datos_completos2 }\OperatorTok\StringTok{ }\KeywordTok{mutate}\NormalTok{(}\DataTypeTok{costos_gastos_totales =}\NormalTok{ gastos_administracion }\OperatorTok{+}\StringTok{ }
\StringTok{                        }\NormalTok{otros_gastos }\OperatorTok{+}\NormalTok{gasto_impuestos }\OperatorTok{+}\StringTok{ }\NormalTok{costo_ventas,}
                        \DataTypeTok{ingresos_totales =}\NormalTok{ ingresos_actividades_ordinarias }\OperatorTok{+}\StringTok{ }
\StringTok{                        }\NormalTok{ingresos_financieros }\OperatorTok{+}\StringTok{ }\NormalTok{otros_ingresos) ->}\StringTok{ }\NormalTok{base_modelado}

\NormalTok{base_modelado <-}\StringTok{ }\KeywordTok{droplevels}\NormalTok{(base_modelado)}
\KeywordTok{summary}\NormalTok{(base_modelado)}
\end{Highlighting}
\end{Shaded}

\begin{verbatim}
##      NIT                                                razon_social    CIIU   
##  Length:82          ANGLOGOLD ASHANTI COLOMBIA S.A.           : 3    B0722:82  
##  Class :character   CALDAS GOLD MARMATO S.A.S.                : 3              
##  Mode  :character   CONTINENTAL GOLD LIMITED SUCURSAL COLOMBIA: 3              
##                     ECO ORO MINERALS CORP                     : 3              
##                     EXPLORACIONES CHAPARRAL COLOMBIA SAS      : 3              
##                     EXPLORACIONES NORTHERN COLOMBIA S.A.S     : 3              
##                     (Other)                                   :64              
##           ciudad          departamento        estado    situacion 
##  BOGOTÁ, D.C.:26   ANTIOQUIA    :44    INSPECCION:41   ACTIVA:82  
##  BUCARAMANGA :12   BOGOTÁ, D. C.:26    VIGILANCIA:41              
##  ENVIGADO    : 2   SANTANDER    :12                               
##  MEDELLÍN    :42                                                  
##                                                                   
##                                                                   
##                                                                   
##                      organo_societario etapa_situacion    Periodo    
##  ACTIVIDAD ECONOMICA DIFERENTE:82      ACTIVA:82       Min.   :2017  
##                                                        1st Qu.:2017  
##                                                        Median :2018  
##                                                        Mean   :2018  
##                                                        3rd Qu.:2019  
##                                                        Max.   :2019  
##                                                                      
##   costo_ventas       gastos_administracion  otros_gastos     
##  Min.   :        0   Min.   :        0     Min.   :       0  
##  1st Qu.:        0   1st Qu.:    15366     1st Qu.:     848  
##  Median :        0   Median :   560711     Median :  120317  
##  Mean   : 23244524   Mean   : 12272119     Mean   : 3275946  
##  3rd Qu.:  6666040   3rd Qu.:  5672411     3rd Qu.: 1736150  
##  Max.   :408474390   Max.   :287863704     Max.   :46689684  
##                                                              
##  costos_financieros gasto_impuestos     ingresos_actividades_ordinarias
##  Min.   :       0   Min.   : -8974634   Min.   :        0              
##  1st Qu.:       0   1st Qu.:        0   1st Qu.:        0              
##  Median :  168546   Median :     2546   Median :        0              
##  Mean   : 2983081   Mean   :  4495562   Mean   : 34205451              
##  3rd Qu.: 2383609   3rd Qu.:   141459   3rd Qu.:        0              
##  Max.   :52689630   Max.   :131684824   Max.   :954650443              
##                                                                        
##  otros_ingresos     ingresos_financieros   TRM_media     TRM_mediana  
##  Min.   :       0   Min.   :       0     Min.   :2951   Min.   :2898  
##  1st Qu.:       0   1st Qu.:       0     1st Qu.:2951   1st Qu.:2898  
##  Median :   36549   Median :    3942     Median :2956   Median :2942  
##  Mean   :  836610   Mean   : 1285604     Mean   :3058   Mean   :3033  
##  3rd Qu.:  601104   3rd Qu.:  850632     3rd Qu.:3281   3rd Qu.:3277  
##  Max.   :11404344   Max.   :26100695     Max.   :3281   Max.   :3277  
##                                                                       
##     PIB_M.E.        Var.PIB        Inflacion       Desempleo     
##  Min.   :255.4   Min.   :1.400   Min.   :3.180   Min.   : 9.400  
##  1st Qu.:255.4   1st Qu.:1.400   1st Qu.:3.180   1st Qu.: 9.400  
##  Median :276.0   Median :2.100   Median :3.800   Median : 9.700  
##  Mean   :270.9   Mean   :1.998   Mean   :3.687   Mean   : 9.851  
##  3rd Qu.:280.2   3rd Qu.:2.500   3rd Qu.:4.090   3rd Qu.:10.500  
##  Max.   :280.2   Max.   :2.500   Max.   :4.090   Max.   :10.500  
##                                                                  
##  var.Desempleo        GNC        Balance_Cuenta_Corriente  TIM_promedio  
##  Min.   :1.990   Min.   :-3.60   Min.   :-13118           Min.   :4.250  
##  1st Qu.:1.990   1st Qu.:-3.60   1st Qu.:-13118           1st Qu.:4.250  
##  Median :3.190   Median :-3.10   Median :-12036           Median :4.350  
##  Mean   :4.385   Mean   :-3.08   Mean   :-11792           Mean   :4.926  
##  3rd Qu.:8.250   3rd Qu.:-2.50   3rd Qu.:-10241           3rd Qu.:6.130  
##  Max.   :8.250   Max.   :-2.50   Max.   :-10241           Max.   :6.130  
##                                                                          
##  costos_gastos_totales ingresos_totales   
##  Min.   :     2869     Min.   :        0  
##  1st Qu.:   528039     1st Qu.:    24364  
##  Median :  4175588     Median :   587346  
##  Mean   : 43288150     Mean   : 36327665  
##  3rd Qu.: 26972121     3rd Qu.:  3698197  
##  Max.   :700560590     Max.   :967119965  
## 
\end{verbatim}

\begin{Shaded}
\begin{Highlighting}[]
\KeywordTok{library}\NormalTok{(visdat)}
\end{Highlighting}
\end{Shaded}

\begin{verbatim}
## Warning: package 'visdat' was built under R version 4.0.3
\end{verbatim}

\begin{Shaded}
\begin{Highlighting}[]
\KeywordTok{vis_miss}\NormalTok{(base_modelado)}
\end{Highlighting}
\end{Shaded}

\includegraphics{index_files/figure-latex/unnamed-chunk-4-1.pdf} Esta
gráfica nos ayuda a visualizar que no hay datos perdidos en la data.

\hypertarget{capuxedtulo-4.-anuxe1lisis_descriptivo}{%
\chapter{Capítulo 4.
Análisis\_descriptivo}\label{capuxedtulo-4.-anuxe1lisis_descriptivo}}

\begin{Shaded}
\begin{Highlighting}[]
\KeywordTok{library}\NormalTok{(ggplot2)}
\KeywordTok{library}\NormalTok{(tidyr)}
\KeywordTok{library}\NormalTok{(dplyr)}

\NormalTok{datosValidar <-}\StringTok{ }\NormalTok{base_modelado }\OperatorTok\StringTok{ }\KeywordTok{group_by}\NormalTok{(NIT, razon_social) }\OperatorTok\StringTok{ }
\StringTok{  }\KeywordTok{summarise}\NormalTok{(}\DataTypeTok{total=}\KeywordTok{n}\NormalTok{())}
\end{Highlighting}
\end{Shaded}

\begin{verbatim}
## `summarise()` regrouping output by 'NIT' (override with `.groups` argument)
\end{verbatim}

\begin{Shaded}
\begin{Highlighting}[]
\NormalTok{h <-}\StringTok{ }\KeywordTok{hist}\NormalTok{(}\DataTypeTok{x =}\NormalTok{ datosValidar}\OperatorTok{$}\NormalTok{total, }\DataTypeTok{main =} \StringTok{"Cantidad de periodos por empresa"}\NormalTok{,}
          \DataTypeTok{xlab =} \StringTok{"Cantidad de periodos"}\NormalTok{, }\DataTypeTok{ylab =} \StringTok{"Empresas"}\NormalTok{, }\DataTypeTok{breaks=}\DecValTok{3}\NormalTok{)}

\KeywordTok{text}\NormalTok{(h}\OperatorTok{$}\NormalTok{mids,h}\OperatorTok{$}\NormalTok{counts,}\DataTypeTok{labels=}\NormalTok{h}\OperatorTok{$}\NormalTok{counts, }\DataTypeTok{adj=}\KeywordTok{c}\NormalTok{(}\FloatTok{0.5}\NormalTok{, }\FloatTok{-0.5}\NormalTok{))}
\end{Highlighting}
\end{Shaded}

\includegraphics{index_files/figure-latex/unnamed-chunk-29-1.pdf}
Podemos observar que hay empresas que no tienen los 3 periodos, solo
trabajaremos con la empresas que tengan los periodos completos.

\begin{Shaded}
\begin{Highlighting}[]
\NormalTok{base_modelado }\OperatorTok\StringTok{ }\KeywordTok{anti_join}\NormalTok{(datosValidar }\OperatorTok\StringTok{ }\KeywordTok{filter}\NormalTok{(total }\OperatorTok{<}\StringTok{ }\DecValTok{3}\NormalTok{) , }\DataTypeTok{by=}\StringTok{"NIT"}\NormalTok{ ) }\OperatorTok\StringTok{ }
\StringTok{  }\KeywordTok{group_by}\NormalTok{(NIT, razon_social) }\OperatorTok\StringTok{ }\KeywordTok{summarise}\NormalTok{(}\DataTypeTok{total=}\KeywordTok{n}\NormalTok{()) }\OperatorTok\StringTok{ }\KeywordTok{arrange}\NormalTok{(}\KeywordTok{desc}\NormalTok{(total))}
\end{Highlighting}
\end{Shaded}

\begin{verbatim}
## `summarise()` regrouping output by 'NIT' (override with `.groups` argument)
\end{verbatim}

\begin{verbatim}
## # A tibble: 24 x 3
## # Groups:   NIT [24]
##    NIT       razon_social                        total
##    <chr>     <fct>                               <int>
##  1 811002172 MINERA CROESUS S.A.S                    3
##  2 830012565 ECO ORO MINERALS CORP                   3
##  3 830127076 ANGLOGOLD ASHANTI COLOMBIA S.A.         3
##  4 860507991 SANTIAGO OIL COMPANY                    3
##  5 890114642 CALDAS GOLD MARMATO S.A.S.              3
##  6 900039998 MINERALES ANDINOS DE OCCIDENTE S.A      3
##  7 900062755 MINERIA INTEGRAL DE COLOMBIA S.A.S.     3
##  8 900063262 SOCIEDAD MINERA DE SANTANDER S.A.S.     3
##  9 900084407 GRAMALOTE COLOMBIA LIMITED              3
## 10 900156833 MINERA DE COBRE QUEBRADONA  SA          3
## # ... with 14 more rows
\end{verbatim}

\begin{Shaded}
\begin{Highlighting}[]
\NormalTok{base_modelado }\OperatorTok\StringTok{ }\KeywordTok{anti_join}\NormalTok{(datosValidar }\OperatorTok\StringTok{ }\KeywordTok{filter}\NormalTok{(total }\OperatorTok{<}\StringTok{ }\DecValTok{3}\NormalTok{) , }\DataTypeTok{by=}\StringTok{"NIT"}\NormalTok{ ) ->}\StringTok{  }
\StringTok{  }\NormalTok{base_modelado}
\NormalTok{base_modelado }\OperatorTok\StringTok{ }\KeywordTok{filter}\NormalTok{(ingresos_totales }\OperatorTok{==}\StringTok{ }\DecValTok{0}\NormalTok{) }\OperatorTok\StringTok{ }\KeywordTok{count}\NormalTok{(NIT) }\OperatorTok\StringTok{ }
\StringTok{  }\KeywordTok{filter}\NormalTok{(n}\OperatorTok{==}\DecValTok{3}\NormalTok{) ->}\StringTok{ }\NormalTok{datosValidar}
\NormalTok{base_modelado }\OperatorTok\StringTok{ }\KeywordTok{anti_join}\NormalTok{(datosValidar, }\DataTypeTok{by=}\StringTok{"NIT"}\NormalTok{ )  ->}\StringTok{  }\NormalTok{base_modelado}

\NormalTok{datosValidar <-}\StringTok{ }\NormalTok{base_modelado }\OperatorTok\StringTok{ }\KeywordTok{group_by}\NormalTok{(NIT, razon_social) }\OperatorTok\StringTok{ }
\StringTok{  }\KeywordTok{summarise}\NormalTok{(}\DataTypeTok{total=}\KeywordTok{n}\NormalTok{())}
\end{Highlighting}
\end{Shaded}

\begin{verbatim}
## `summarise()` regrouping output by 'NIT' (override with `.groups` argument)
\end{verbatim}

\begin{Shaded}
\begin{Highlighting}[]
\KeywordTok{table}\NormalTok{(datosValidar}\OperatorTok{$}\NormalTok{total)}
\end{Highlighting}
\end{Shaded}

\begin{verbatim}
## 
##  3 
## 23
\end{verbatim}

Ya podemos ver que tenemos 23 empresas con los 3 periodos. Veamos los
ingresos y los costos por departamento.

\begin{Shaded}
\begin{Highlighting}[]
\NormalTok{datosValidarDepartamento <-}\StringTok{ }\NormalTok{base_modelado }\OperatorTok\StringTok{ }\KeywordTok{group_by}\NormalTok{(departamento, Periodo) }\OperatorTok\StringTok{ }
\StringTok{  }\KeywordTok{summarise}\NormalTok{(}\DataTypeTok{costo_gasto_total_dep =} \KeywordTok{sum}\NormalTok{(costos_gastos_totales) }\OperatorTok{/}\StringTok{ }\DecValTok{1000000}\NormalTok{,}
  \DataTypeTok{ingresos_totales_dep =} \KeywordTok{sum}\NormalTok{(ingresos_totales) }\OperatorTok{/}\StringTok{ }\DecValTok{1000000}\NormalTok{)}
\end{Highlighting}
\end{Shaded}

\begin{verbatim}
## `summarise()` regrouping output by 'departamento' (override with `.groups` argument)
\end{verbatim}

\begin{Shaded}
\begin{Highlighting}[]
\KeywordTok{ggplot}\NormalTok{(datosValidarDepartamento, }\KeywordTok{aes}\NormalTok{(}\DataTypeTok{x=}\NormalTok{ Periodo))}\OperatorTok{+}
\StringTok{  }\KeywordTok{geom_line}\NormalTok{(}\KeywordTok{aes}\NormalTok{(}\DataTypeTok{y =}\NormalTok{ costo_gasto_total_dep), }\DataTypeTok{color=}\StringTok{"darkred"}\NormalTok{, }\DataTypeTok{linetype=}\StringTok{"twodash"}\NormalTok{)}\OperatorTok{+}
\StringTok{  }\KeywordTok{geom_label}\NormalTok{(}\KeywordTok{aes}\NormalTok{(}\DataTypeTok{y =}\NormalTok{ costo_gasto_total_dep, }\DataTypeTok{label=}\NormalTok{costo_gasto_total_dep)) }\OperatorTok{+}\StringTok{ }
\StringTok{  }\KeywordTok{geom_line}\NormalTok{(}\KeywordTok{aes}\NormalTok{(}\DataTypeTok{y =}\NormalTok{ ingresos_totales_dep, }\DataTypeTok{label=}\StringTok{"Ingresos"}\NormalTok{), }\DataTypeTok{color =} \StringTok{"steelblue"}\NormalTok{)}\OperatorTok{+}
\StringTok{  }\KeywordTok{geom_label}\NormalTok{(}\KeywordTok{aes}\NormalTok{(}\DataTypeTok{y =}\NormalTok{ ingresos_totales_dep, }\DataTypeTok{label=}\NormalTok{ingresos_totales_dep)) }\OperatorTok{+}\StringTok{ }
\StringTok{  }\KeywordTok{facet_wrap}\NormalTok{(}\OperatorTok{~}\NormalTok{departamento, }\DataTypeTok{scales =}\StringTok{"free_y"}\NormalTok{)}
\end{Highlighting}
\end{Shaded}

\begin{verbatim}
## Warning: Ignoring unknown aesthetics: label
\end{verbatim}

\includegraphics{index_files/figure-latex/unnamed-chunk-31-1.pdf}

\begin{Shaded}
\begin{Highlighting}[]
\NormalTok{base_modelado}\OperatorTok{$}\NormalTok{NIT=}\KeywordTok{as.factor}\NormalTok{(base_modelado}\OperatorTok{$}\NormalTok{NIT)}
\NormalTok{base_modelado}\OperatorTok{$}\NormalTok{Periodo=}\KeywordTok{as.factor}\NormalTok{(base_modelado}\OperatorTok{$}\NormalTok{Periodo)}

\NormalTok{p1=}\KeywordTok{ggplot}\NormalTok{(base_modelado, }\KeywordTok{aes}\NormalTok{(}\DataTypeTok{y=}\NormalTok{costos_gastos_totales,}\DataTypeTok{x=}\NormalTok{Periodo,}\DataTypeTok{group=}\NormalTok{NIT,}\DataTypeTok{colour=}\NormalTok{departamento))}
\NormalTok{p1}\OperatorTok{+}\KeywordTok{geom_line}\NormalTok{()}
\end{Highlighting}
\end{Shaded}

\includegraphics{index_files/figure-latex/unnamed-chunk-32-1.pdf} Se
pueden ver algunos comportamientos diferentes por departamento, sin
embargo separemos el gráfico para ver mejor:

\begin{Shaded}
\begin{Highlighting}[]
\NormalTok{p1}\OperatorTok{+}\KeywordTok{geom_line}\NormalTok{()}\OperatorTok{+}\KeywordTok{facet_grid}\NormalTok{(.}\OperatorTok{~}\NormalTok{departamento)}
\end{Highlighting}
\end{Shaded}

\includegraphics{index_files/figure-latex/unnamed-chunk-33-1.pdf} El
gráfico anterior nos muestra que cada empresa tiene costos/gastos
totales particulares. Adicionalmente, hay una empresa de Medellín que
tiene costos/gastos totales mas altos, comparada con las otras. Tratemos
de identificar las empresas que tienen un comportamiento más diferente a
las demás.

\begin{Shaded}
\begin{Highlighting}[]
\KeywordTok{theme_set}\NormalTok{(}\KeywordTok{theme_bw}\NormalTok{(}\DataTypeTok{base_size =} \DecValTok{8}\NormalTok{))}

\KeywordTok{qplot}\NormalTok{(NIT, costos_gastos_totales, }\DataTypeTok{facets =}\NormalTok{ . }\OperatorTok{~}\StringTok{ }\NormalTok{departamento, }
      \DataTypeTok{colour =}\NormalTok{ NIT, }\DataTypeTok{geom =} \StringTok{"boxplot"}\NormalTok{, }\DataTypeTok{data =}\NormalTok{ base_modelado)}
\end{Highlighting}
\end{Shaded}

\includegraphics{index_files/figure-latex/unnamed-chunk-34-1.pdf} Al
parecer solo hay 1 empresa que tiene comportamiento de costos/gastos
totales mucho mas diferente a las demás.

Realizaremos el ejercicio de eliminar (solo para efectos visuales) la
empresa que es mas diferente a las demas.

\begin{Shaded}
\begin{Highlighting}[]
\NormalTok{datosValidar <-}\StringTok{ }\KeywordTok{filter}\NormalTok{(base_modelado, NIT}\OperatorTok{!=}\DecValTok{900306309}\NormalTok{)}
\NormalTok{p1=}\KeywordTok{ggplot}\NormalTok{(datosValidar, }\KeywordTok{aes}\NormalTok{(}\DataTypeTok{y=}\NormalTok{costos_gastos_totales}\OperatorTok{/}\DecValTok{1000}\NormalTok{,}\DataTypeTok{x=}\NormalTok{Periodo,}\DataTypeTok{group=}\NormalTok{NIT,}
                            \DataTypeTok{colour=}\NormalTok{departamento))}
\NormalTok{p1}\OperatorTok{+}\KeywordTok{geom_line}\NormalTok{()}\OperatorTok{+}\KeywordTok{facet_grid}\NormalTok{(.}\OperatorTok{~}\NormalTok{departamento)}
\end{Highlighting}
\end{Shaded}

\includegraphics{index_files/figure-latex/unnamed-chunk-35-1.pdf}
Confirmamos que los costos/gastos totales son particulares de cada
empresa. Cambiemos la escala de los datos y volvamos a graficar, para
poder apreciar mejor el comportamiento de las otras empresas que tienen
costos/gastos totales mas bajos, pero con el set de empresas completo.

\begin{Shaded}
\begin{Highlighting}[]
\KeywordTok{theme_set}\NormalTok{(}\KeywordTok{theme_bw}\NormalTok{(}\DataTypeTok{base_size =} \DecValTok{8}\NormalTok{))}

\KeywordTok{qplot}\NormalTok{(NIT, costos_gastos_totales, }\DataTypeTok{facets =}\NormalTok{ . }\OperatorTok{~}\StringTok{ }\NormalTok{departamento, }
      \DataTypeTok{colour =}\NormalTok{ NIT, }\DataTypeTok{geom =} \StringTok{"boxplot"}\NormalTok{, }\DataTypeTok{data =}\NormalTok{ base_modelado) }\OperatorTok{+}
\StringTok{  }\KeywordTok{scale_y_log10}\NormalTok{() }\OperatorTok{+}\StringTok{ }
\StringTok{  }\KeywordTok{geom_hline}\NormalTok{(}\KeywordTok{aes}\NormalTok{(}\DataTypeTok{yintercept =} \KeywordTok{mean}\NormalTok{(costos_gastos_totales)), }\DataTypeTok{color =} \StringTok{"steelblue"}\NormalTok{) }\OperatorTok{+}
\StringTok{  }
\StringTok{  }\KeywordTok{geom_hline}\NormalTok{(}\KeywordTok{aes}\NormalTok{(}\DataTypeTok{yintercept =} \KeywordTok{median}\NormalTok{(costos_gastos_totales)), }\DataTypeTok{color =} \StringTok{"red"}\NormalTok{)}
\end{Highlighting}
\end{Shaded}

\includegraphics{index_files/figure-latex/unnamed-chunk-36-1.pdf} Ahora
podemos ver mejor que cada empresa tiene unos costos/gastos totales
particulares, así como costos promedio diferentes. Además, encontramos
que solamente hay 5 empresas que tienen un comportamiento general en sus
costos/gastos totales. Ahora revisemos los costos/gastos totales con el
estado.

\begin{Shaded}
\begin{Highlighting}[]
\KeywordTok{theme_set}\NormalTok{(}\KeywordTok{theme_bw}\NormalTok{(}\DataTypeTok{base_size =} \DecValTok{8}\NormalTok{))}

\KeywordTok{qplot}\NormalTok{(NIT, costos_gastos_totales}\OperatorTok{/}\DecValTok{1000}\NormalTok{, }\DataTypeTok{facets =}\NormalTok{ . }\OperatorTok{~}\StringTok{ }\NormalTok{estado, }
      \DataTypeTok{colour =}\NormalTok{ NIT, }\DataTypeTok{geom =} \StringTok{"boxplot"}\NormalTok{, }\DataTypeTok{data =}\NormalTok{ base_modelado) }
\end{Highlighting}
\end{Shaded}

\includegraphics{index_files/figure-latex/unnamed-chunk-37-1.pdf}

Podemos ver que las empresas con estado inspección presentan
costos/gastos totales menores que las empresas con estado vigilancia.

Veamos ahora la dispersion de nuestra variable objetivo.

\begin{Shaded}
\begin{Highlighting}[]
\KeywordTok{plot}\NormalTok{(base_modelado}\OperatorTok{$}\NormalTok{costos_gastos_totales, }\DataTypeTok{main=}\StringTok{"Costos/gastos totales"}\NormalTok{,}
    \DataTypeTok{type=}\StringTok{"b"}\NormalTok{, }\DataTypeTok{ylab=}\StringTok{"Costos/gastos totales/10000"}\NormalTok{, }\DataTypeTok{pch=} \DecValTok{20}\NormalTok{, }\DataTypeTok{lwd=}\DecValTok{2}\NormalTok{)}
\KeywordTok{abline}\NormalTok{(}\DataTypeTok{h=}\KeywordTok{mean}\NormalTok{(base_modelado}\OperatorTok{$}\NormalTok{costos_gastos_totales), }\DataTypeTok{lwd=}\DecValTok{2}\NormalTok{, }\DataTypeTok{col=} \StringTok{"red"}\NormalTok{)}
\KeywordTok{grid}\NormalTok{()}
\end{Highlighting}
\end{Shaded}

\includegraphics{index_files/figure-latex/unnamed-chunk-38-1.pdf} Con
esto confirmamos que la dispersion de los costos/gastos totales no tiene
un comportamiento general.

\hypertarget{capuxedtulo-5.-correlaciones}{%
\chapter{Capítulo 5. Correlaciones}\label{capuxedtulo-5.-correlaciones}}

\begin{verbatim}
## Warning: package 'Hmisc' was built under R version 4.0.3
\end{verbatim}

\begin{verbatim}
## Loading required package: lattice
\end{verbatim}

\begin{verbatim}
## Loading required package: survival
\end{verbatim}

\begin{verbatim}
## Loading required package: Formula
\end{verbatim}

\begin{verbatim}
## Warning: package 'Formula' was built under R version 4.0.3
\end{verbatim}

\begin{verbatim}
## 
## Attaching package: 'Hmisc'
\end{verbatim}

\begin{verbatim}
## The following objects are masked from 'package:dplyr':
## 
##     src, summarize
\end{verbatim}

\begin{verbatim}
## The following objects are masked from 'package:base':
## 
##     format.pval, units
\end{verbatim}

\begin{verbatim}
## Warning: package 'corrplot' was built under R version 4.0.3
\end{verbatim}

\begin{verbatim}
## corrplot 0.84 loaded
\end{verbatim}

\begin{verbatim}
## Warning: package 'PerformanceAnalytics' was built under R version 4.0.3
\end{verbatim}

\begin{verbatim}
## Loading required package: xts
\end{verbatim}

\begin{verbatim}
## Warning: package 'xts' was built under R version 4.0.3
\end{verbatim}

\begin{verbatim}
## Loading required package: zoo
\end{verbatim}

\begin{verbatim}
## Warning: package 'zoo' was built under R version 4.0.3
\end{verbatim}

\begin{verbatim}
## 
## Attaching package: 'zoo'
\end{verbatim}

\begin{verbatim}
## The following objects are masked from 'package:base':
## 
##     as.Date, as.Date.numeric
\end{verbatim}

\begin{verbatim}
## 
## Attaching package: 'xts'
\end{verbatim}

\begin{verbatim}
## The following objects are masked from 'package:dplyr':
## 
##     first, last
\end{verbatim}

\begin{verbatim}
## 
## Attaching package: 'PerformanceAnalytics'
\end{verbatim}

\begin{verbatim}
## The following object is masked from 'package:graphics':
## 
##     legend
\end{verbatim}

Se presenta la base de datos

\begin{Shaded}
\begin{Highlighting}[]
\KeywordTok{head}\NormalTok{(base_modelado)}
\end{Highlighting}
\end{Shaded}

\begin{verbatim}
##         NIT                       razon_social  CIIU       ciudad  departamento
## 1 811002172               MINERA CROESUS S.A.S B0722     MEDELLÍN     ANTIOQUIA
## 2 830012565              ECO ORO MINERALS CORP B0722  BUCARAMANGA     SANTANDER
## 3 830127076    ANGLOGOLD ASHANTI COLOMBIA S.A. B0722 BOGOTÁ, D.C. BOGOTÁ, D. C.
## 4 860507991               SANTIAGO OIL COMPANY B0722 BOGOTÁ, D.C. BOGOTÁ, D. C.
## 5 890114642         CALDAS GOLD MARMATO S.A.S. B0722     MEDELLÍN     ANTIOQUIA
## 6 900039998 MINERALES ANDINOS DE OCCIDENTE S.A B0722     MEDELLÍN     ANTIOQUIA
##       estado situacion             organo_societario etapa_situacion Periodo
## 1 INSPECCION    ACTIVA ACTIVIDAD ECONOMICA DIFERENTE          ACTIVA    2019
## 2 INSPECCION    ACTIVA ACTIVIDAD ECONOMICA DIFERENTE          ACTIVA    2019
## 3 VIGILANCIA    ACTIVA ACTIVIDAD ECONOMICA DIFERENTE          ACTIVA    2019
## 4 VIGILANCIA    ACTIVA ACTIVIDAD ECONOMICA DIFERENTE          ACTIVA    2019
## 5 VIGILANCIA    ACTIVA ACTIVIDAD ECONOMICA DIFERENTE          ACTIVA    2019
## 6 INSPECCION    ACTIVA ACTIVIDAD ECONOMICA DIFERENTE          ACTIVA    2019
##   costo_ventas gastos_administracion otros_gastos costos_financieros
## 1            0                 53823            0            3271606
## 2            0               6963079     15986104                  0
## 3     35455996                     0            0            2437764
## 4     21785731                920119       418332            4972993
## 5     95446409               2372895       621250            2425250
## 6            0                     0      1053083            7160223
##   gasto_impuestos ingresos_actividades_ordinarias otros_ingresos
## 1           18725                               0         975059
## 2           49834                               0         565503
## 3               0                               0              0
## 4        -6062218                        27158134         274473
## 5         6473853                       117246642        1691325
## 6            2414                               0              0
##   ingresos_financieros TRM_media TRM_mediana PIB_M.E. Var.PIB Inflacion
## 1                    0  3281.092     3277.14  255.416     2.1       3.8
## 2                    0  3281.092     3277.14  255.416     2.1       3.8
## 3                    0  3281.092     3277.14  255.416     2.1       3.8
## 4               885317  3281.092     3277.14  255.416     2.1       3.8
## 5              3104595  3281.092     3277.14  255.416     2.1       3.8
## 6                    0  3281.092     3277.14  255.416     2.1       3.8
##   Desempleo var.Desempleo  GNC Balance_Cuenta_Corriente TIM_promedio
## 1      10.5          8.25 -2.5                -12036.18         4.25
## 2      10.5          8.25 -2.5                -12036.18         4.25
## 3      10.5          8.25 -2.5                -12036.18         4.25
## 4      10.5          8.25 -2.5                -12036.18         4.25
## 5      10.5          8.25 -2.5                -12036.18         4.25
## 6      10.5          8.25 -2.5                -12036.18         4.25
##   costos_gastos_totales ingresos_totales
## 1                 72548           975059
## 2              22999017           565503
## 3              35455996                0
## 4              17061964         28317924
## 5             104914407        122042562
## 6               1055497                0
\end{verbatim}

Para el análisis de correlaciones se toman las variables macroeconomicas
(PIB, Inflación, Desempleo, GNC, Balance de cuenta corriente y TIM) y se
comparan respecto a los costos de ventas.

\begin{Shaded}
\begin{Highlighting}[]
\CommentTok{#A continuación se agrupan las variables de interés en un nuevo dataframe}

\NormalTok{base =}\StringTok{ }\NormalTok{base_modelado[,}\DecValTok{19}\OperatorTok{:}\DecValTok{28}\NormalTok{]}
\NormalTok{base[}\StringTok{'Var.objetivo'}\NormalTok{]=}\StringTok{ }\NormalTok{base_modelado[}\DecValTok{29}\NormalTok{]}
\KeywordTok{head}\NormalTok{(base)}
\end{Highlighting}
\end{Shaded}

\begin{verbatim}
##   TRM_media TRM_mediana PIB_M.E. Var.PIB Inflacion Desempleo var.Desempleo  GNC
## 1  3281.092     3277.14  255.416     2.1       3.8      10.5          8.25 -2.5
## 2  3281.092     3277.14  255.416     2.1       3.8      10.5          8.25 -2.5
## 3  3281.092     3277.14  255.416     2.1       3.8      10.5          8.25 -2.5
## 4  3281.092     3277.14  255.416     2.1       3.8      10.5          8.25 -2.5
## 5  3281.092     3277.14  255.416     2.1       3.8      10.5          8.25 -2.5
## 6  3281.092     3277.14  255.416     2.1       3.8      10.5          8.25 -2.5
##   Balance_Cuenta_Corriente TIM_promedio Var.objetivo
## 1                -12036.18         4.25        72548
## 2                -12036.18         4.25     22999017
## 3                -12036.18         4.25     35455996
## 4                -12036.18         4.25     17061964
## 5                -12036.18         4.25    104914407
## 6                -12036.18         4.25      1055497
\end{verbatim}

\#Calcular el coeficiente de correlación Este comando calcula la matriz
de correlación:

\begin{Shaded}
\begin{Highlighting}[]
\KeywordTok{round}\NormalTok{(}\KeywordTok{cor}\NormalTok{(base),}\DecValTok{2}\NormalTok{)  }
\end{Highlighting}
\end{Shaded}

\begin{verbatim}
##                          TRM_media TRM_mediana PIB_M.E. Var.PIB Inflacion
## TRM_media                     1.00        0.99    -0.99    0.14      0.19
## TRM_mediana                   0.99        1.00    -0.96    0.26      0.31
## PIB_M.E.                     -0.99       -0.96     1.00    0.00     -0.05
## Var.PIB                       0.14        0.26     0.00    1.00      1.00
## Inflacion                     0.19        0.31    -0.05    1.00      1.00
## Desempleo                     0.97        0.93    -0.99   -0.11     -0.06
## var.Desempleo                 0.99        0.96    -1.00   -0.03      0.02
## GNC                           0.90        0.84    -0.95   -0.31     -0.26
## Balance_Cuenta_Corriente     -0.13       -0.25    -0.02   -1.00     -1.00
## TIM_promedio                 -0.55       -0.45     0.67    0.75      0.71
## Var.objetivo                  0.04        0.04    -0.04    0.00      0.01
##                          Desempleo var.Desempleo   GNC Balance_Cuenta_Corriente
## TRM_media                     0.97          0.99  0.90                    -0.13
## TRM_mediana                   0.93          0.96  0.84                    -0.25
## PIB_M.E.                     -0.99         -1.00 -0.95                    -0.02
## Var.PIB                      -0.11         -0.03 -0.31                    -1.00
## Inflacion                    -0.06          0.02 -0.26                    -1.00
## Desempleo                     1.00          1.00  0.98                     0.12
## var.Desempleo                 1.00          1.00  0.96                     0.04
## GNC                           0.98          0.96  1.00                     0.32
## Balance_Cuenta_Corriente      0.12          0.04  0.32                     1.00
## TIM_promedio                 -0.74         -0.68 -0.86                    -0.76
## Var.objetivo                  0.04          0.04  0.03                     0.00
##                          TIM_promedio Var.objetivo
## TRM_media                       -0.55         0.04
## TRM_mediana                     -0.45         0.04
## PIB_M.E.                         0.67        -0.04
## Var.PIB                          0.75         0.00
## Inflacion                        0.71         0.01
## Desempleo                       -0.74         0.04
## var.Desempleo                   -0.68         0.04
## GNC                             -0.86         0.03
## Balance_Cuenta_Corriente        -0.76         0.00
## TIM_promedio                     1.00        -0.02
## Var.objetivo                    -0.02         1.00
\end{verbatim}

Podemos interpretar que, la correlación entre las varianbles
macroeconomicas y la variable objetivo no son explicativas, si nivel de
significancia es cercano a cero. Es decir, no hay una ascociación entre
estas variables y la variable objetivo, que nos ayude a predecir o
explicar el comportamiento de los costos y gastos totales.

\#Ver la matriz de forma gráfica Podemos graficar con el comando
corrplot. Ver más en este enlace: Lo primero es calcular la matriz de
correlación y guardarla en un objeto y luego graficarlo. En este caso
vamos a graficar los coeficientes.

\begin{Shaded}
\begin{Highlighting}[]
\NormalTok{correlacion<-}\KeywordTok{round}\NormalTok{(}\KeywordTok{cor}\NormalTok{(base), }\DecValTok{1}\NormalTok{)}

\KeywordTok{corrplot}\NormalTok{(correlacion, }\DataTypeTok{method=}\StringTok{"number"}\NormalTok{, }\DataTypeTok{type=}\StringTok{"upper"}\NormalTok{)}
\end{Highlighting}
\end{Shaded}

\includegraphics{index_files/figure-latex/unnamed-chunk-43-1.pdf} A
continuación se grafican los datos de la Variable objetivo (costos y
gastos totales) con respecto a var.PIB (Variación del PIB)

\begin{Shaded}
\begin{Highlighting}[]
\KeywordTok{ggplot}\NormalTok{(base, }\KeywordTok{aes}\NormalTok{(}\DataTypeTok{x=}\NormalTok{Var.PIB, }\DataTypeTok{y=}\NormalTok{Var.objetivo)) }\OperatorTok{+}\KeywordTok{geom_point}\NormalTok{()}\OperatorTok{+}\KeywordTok{scale_y_log10}\NormalTok{()}
\end{Highlighting}
\end{Shaded}

\includegraphics{index_files/figure-latex/unnamed-chunk-44-1.pdf}
Podemos apreciar que no se presenta un comportamiento lineal entre las
variables. Por tal motivo se decide no trabajar con las variables
macroeconomicas en los modelos.

\begin{Shaded}
\begin{Highlighting}[]
\NormalTok{base_modelado }\OperatorTok\StringTok{ }\KeywordTok{select}\NormalTok{(departamento, estado, Periodo, costos_gastos_totales,}
\NormalTok{  ciudad, ingresos_totales, NIT, razon_social,otros_ingresos ,}
\NormalTok{  ingresos_financieros) ->}\StringTok{ }\NormalTok{base_modelo_lineal}

\CommentTok{#definir variable de tamaño de la empresa}
\NormalTok{bussiness_size <-}\StringTok{ }\KeywordTok{cut}\NormalTok{(base_modelo_lineal}\OperatorTok{$}\NormalTok{costos_gastos_totales, }\DataTypeTok{breaks=}\DecValTok{4}\NormalTok{)}
\KeywordTok{levels}\NormalTok{(bussiness_size) <-}\StringTok{ }\KeywordTok{list}\NormalTok{(}\DataTypeTok{small =} \StringTok{"(-6.98e+05,1.75e+08]"}\NormalTok{, }
                               \DataTypeTok{medium =} \StringTok{"(1.75e+08,3.5e+08]"}\NormalTok{, }
                               \DataTypeTok{big =} \StringTok{"(3.5e+08,5.25e+08]"}\NormalTok{, }
                               \DataTypeTok{very_big=}\StringTok{"(5.25e+08,7.01e+08]"}\NormalTok{)}
\NormalTok{base_modelo_lineal[}\StringTok{'tamano_empresa'}\NormalTok{] <-}\StringTok{ }\NormalTok{bussiness_size}
\NormalTok{base_modelo_lineal }\OperatorTok\StringTok{ }\KeywordTok{filter}\NormalTok{(NIT }\OperatorTok{!=}\StringTok{ }\DecValTok{900306309}\NormalTok{) ->}\StringTok{ }\NormalTok{base_modelo_lineal}
\KeywordTok{head}\NormalTok{(base_modelo_lineal)}
\end{Highlighting}
\end{Shaded}

\begin{verbatim}
##    departamento     estado Periodo costos_gastos_totales       ciudad
## 1     ANTIOQUIA INSPECCION    2019                 72548     MEDELLÍN
## 2     SANTANDER INSPECCION    2019              22999017  BUCARAMANGA
## 3 BOGOTÁ, D. C. VIGILANCIA    2019              35455996 BOGOTÁ, D.C.
## 4 BOGOTÁ, D. C. VIGILANCIA    2019              17061964 BOGOTÁ, D.C.
## 5     ANTIOQUIA VIGILANCIA    2019             104914407     MEDELLÍN
## 6     ANTIOQUIA INSPECCION    2019               1055497     MEDELLÍN
##   ingresos_totales       NIT                       razon_social otros_ingresos
## 1           975059 811002172               MINERA CROESUS S.A.S         975059
## 2           565503 830012565              ECO ORO MINERALS CORP         565503
## 3                0 830127076    ANGLOGOLD ASHANTI COLOMBIA S.A.              0
## 4         28317924 860507991               SANTIAGO OIL COMPANY         274473
## 5        122042562 890114642         CALDAS GOLD MARMATO S.A.S.        1691325
## 6                0 900039998 MINERALES ANDINOS DE OCCIDENTE S.A              0
##   ingresos_financieros tamano_empresa
## 1                    0          small
## 2                    0          small
## 3                    0          small
## 4               885317          small
## 5              3104595          small
## 6                    0          small
\end{verbatim}

\hypertarget{capuxedtulo-6.-aplicaciuxf3n-de-modelo}{%
\chapter{Capítulo 6. Aplicación de
modelo}\label{capuxedtulo-6.-aplicaciuxf3n-de-modelo}}

\hypertarget{modelo-regresiuxf3n-lineal}{%
\subsection{6.1. Modelo regresión
Lineal}\label{modelo-regresiuxf3n-lineal}}

\begin{Shaded}
\begin{Highlighting}[]
\KeywordTok{library}\NormalTok{(broom)}

\NormalTok{mod1 <-}\StringTok{ }\KeywordTok{lm}\NormalTok{(costos_gastos_totales }\OperatorTok{~}\StringTok{  }\NormalTok{estado, }\DataTypeTok{data=}\NormalTok{ base_modelo_lineal) }
\KeywordTok{anova}\NormalTok{(mod1)}
\end{Highlighting}
\end{Shaded}

\begin{verbatim}
## Analysis of Variance Table
## 
## Response: costos_gastos_totales
##           Df     Sum Sq    Mean Sq F value    Pr(>F)    
## estado     1 1.6640e+16 1.6640e+16  30.862 5.742e-07 ***
## Residuals 64 3.4507e+16 5.3918e+14                      
## ---
## Signif. codes:  0 '***' 0.001 '**' 0.01 '*' 0.05 '.' 0.1 ' ' 1
\end{verbatim}

calculamos el resumen del modelo 1

\begin{Shaded}
\begin{Highlighting}[]
\KeywordTok{summary}\NormalTok{(mod1)}
\end{Highlighting}
\end{Shaded}

\begin{verbatim}
## 
## Call:
## lm(formula = costos_gastos_totales ~ estado, data = base_modelo_lineal)
## 
## Residuals:
##       Min        1Q    Median        3Q       Max 
## -34469500 -10761423  -2118984   3368677  70203031 
## 
## Coefficients:
##                  Estimate Std. Error t value Pr(>|t|)    
## (Intercept)       2954601    4042112   0.731    0.467    
## estadoVIGILANCIA 31756775    5716409   5.555 5.74e-07 ***
## ---
## Signif. codes:  0 '***' 0.001 '**' 0.01 '*' 0.05 '.' 0.1 ' ' 1
## 
## Residual standard error: 23220000 on 64 degrees of freedom
## Multiple R-squared:  0.3253, Adjusted R-squared:  0.3148 
## F-statistic: 30.86 on 1 and 64 DF,  p-value: 5.742e-07
\end{verbatim}

\hypertarget{modelo-regresiuxf3n-lineal-sin-efectos-aleatorios}{%
\subsection{6.2. Modelo Regresión lineal Sin efectos
aleatorios}\label{modelo-regresiuxf3n-lineal-sin-efectos-aleatorios}}

\begin{Shaded}
\begin{Highlighting}[]
\NormalTok{mod2 <-}\StringTok{ }\KeywordTok{lm}\NormalTok{(costos_gastos_totales }\OperatorTok{~}\StringTok{ }\NormalTok{estado }\OperatorTok{+}\StringTok{ }\NormalTok{ingresos_totales, }
           \DataTypeTok{data=}\NormalTok{ base_modelo_lineal) }
\KeywordTok{anova}\NormalTok{(mod2)}
\end{Highlighting}
\end{Shaded}

\begin{verbatim}
## Analysis of Variance Table
## 
## Response: costos_gastos_totales
##                  Df     Sum Sq    Mean Sq F value    Pr(>F)    
## estado            1 1.6640e+16 1.6640e+16  51.004 1.131e-09 ***
## ingresos_totales  1 1.3953e+16 1.3953e+16  42.769 1.258e-08 ***
## Residuals        63 2.0554e+16 3.2625e+14                      
## ---
## Signif. codes:  0 '***' 0.001 '**' 0.01 '*' 0.05 '.' 0.1 ' ' 1
\end{verbatim}

\begin{Shaded}
\begin{Highlighting}[]
\KeywordTok{summary}\NormalTok{(mod2)}
\end{Highlighting}
\end{Shaded}

\begin{verbatim}
## 
## Call:
## lm(formula = costos_gastos_totales ~ estado + ingresos_totales, 
##     data = base_modelo_lineal)
## 
## Residuals:
##       Min        1Q    Median        3Q       Max 
## -30255617  -4992649  -1880040   2979397  55986413 
## 
## Coefficients:
##                   Estimate Std. Error t value Pr(>|t|)    
## (Intercept)      2.471e+06  3.145e+06   0.786    0.435    
## estadoVIGILANCIA 2.231e+07  4.675e+06   4.773 1.12e-05 ***
## ingresos_totales 6.869e-01  1.050e-01   6.540 1.26e-08 ***
## ---
## Signif. codes:  0 '***' 0.001 '**' 0.01 '*' 0.05 '.' 0.1 ' ' 1
## 
## Residual standard error: 18060000 on 63 degrees of freedom
## Multiple R-squared:  0.5981, Adjusted R-squared:  0.5854 
## F-statistic: 46.89 on 2 and 63 DF,  p-value: 3.375e-13
\end{verbatim}

\hypertarget{modelo-lineal-con-intercepto-aleatorio}{%
\subsection{6.3. Modelo lineal con intercepto
aleatorio}\label{modelo-lineal-con-intercepto-aleatorio}}

\begin{Shaded}
\begin{Highlighting}[]
\KeywordTok{library}\NormalTok{(lme4)}
\end{Highlighting}
\end{Shaded}

\begin{verbatim}
## Warning: package 'lme4' was built under R version 4.0.3
\end{verbatim}

\begin{verbatim}
## Loading required package: Matrix
\end{verbatim}

\begin{verbatim}
## 
## Attaching package: 'Matrix'
\end{verbatim}

\begin{verbatim}
## The following objects are masked from 'package:tidyr':
## 
##     expand, pack, unpack
\end{verbatim}

\begin{Shaded}
\begin{Highlighting}[]
\NormalTok{mod4 <-}\StringTok{ }\KeywordTok{lmer}\NormalTok{(costos_gastos_totales }\OperatorTok{~}\StringTok{ }\NormalTok{ingresos_totales }\OperatorTok{+}\StringTok{ }\NormalTok{(}\DecValTok{1}\OperatorTok{|}\NormalTok{departamento), }
             \DataTypeTok{data=}\NormalTok{ base_modelo_lineal) }
\end{Highlighting}
\end{Shaded}

\begin{verbatim}
## Warning: Some predictor variables are on very different scales: consider
## rescaling
\end{verbatim}

\begin{verbatim}
## boundary (singular) fit: see ?isSingular
\end{verbatim}

\begin{Shaded}
\begin{Highlighting}[]
\KeywordTok{anova}\NormalTok{(mod4)}
\end{Highlighting}
\end{Shaded}

\begin{verbatim}
## Analysis of Variance Table
##                  npar     Sum Sq    Mean Sq F value
## ingresos_totales    1 2.3162e+16 2.3162e+16  52.967
\end{verbatim}

\begin{Shaded}
\begin{Highlighting}[]
\KeywordTok{summary}\NormalTok{(mod4)}
\end{Highlighting}
\end{Shaded}

\begin{verbatim}
## Linear mixed model fit by REML ['lmerMod']
## Formula: costos_gastos_totales ~ ingresos_totales + (1 | departamento)
##    Data: base_modelo_lineal
## 
## REML criterion at convergence: 2381.4
## 
## Scaled residuals: 
##     Min      1Q  Median      3Q     Max 
## -1.0611 -0.5900 -0.4911  0.3066  3.2155 
## 
## Random effects:
##  Groups       Name        Variance  Std.Dev. 
##  departamento (Intercept) 1.290e-05 3.592e-03
##  Residual                 4.373e+14 2.091e+07
## Number of obs: 66, groups:  departamento, 3
## 
## Fixed effects:
##                   Estimate Std. Error t value
## (Intercept)      1.246e+07  2.719e+06   4.581
## ingresos_totales 8.417e-01  1.157e-01   7.278
## 
## Correlation of Fixed Effects:
##             (Intr)
## ingrss_ttls -0.322
## fit warnings:
## Some predictor variables are on very different scales: consider rescaling
## convergence code: 0
## boundary (singular) fit: see ?isSingular
\end{verbatim}

\hypertarget{capuxedtulo-7.-estimaciuxf3n-de-esfuerzo}{%
\chapter{Capítulo 7. Estimación de
esfuerzo}\label{capuxedtulo-7.-estimaciuxf3n-de-esfuerzo}}

Para las actividades se realiza la siguiente estimación de esfuerzo:

\begin{enumerate}
\def\labelenumi{\arabic{enumi})}
\tightlist
\item
  Consolidación de información: 14h
\item
  Transformación de varibles y análisis descriptivo: 7h
\item
  Ajuste y validación de modelos 8h
\item
  Redacción del reporte: 8h
\end{enumerate}

\hypertarget{capuxedtulo-8.-conclusiones}{%
\chapter{Capítulo 8. Conclusiones}\label{capuxedtulo-8.-conclusiones}}

Se verificaron tres tipos de modelos el primero con una sola variable
explicativa llamada \texttt{estado} el cual tiene un aporte al costo y
al gasto de forma positiva, el segundo modelo se encuentra que agregando
la variable de ingresos totales y el estado el modelo un mejor ajuste de
explicación del gasto y el costo para el sector minero. Para el tercer
modelo que es el que tiene \texttt{departamento} evidenciamos que no es
un modelo apropiado para predecir los costos y gastos del sector minero
trabajado, debido a que las variables no son significativas. Por lo
anterior se selecciona como un posible modelo el modelo
\texttt{número\ 2\ sin\ efecto\ aleatorio} el cual presenta el menor
residual.

\hypertarget{referencias}{%
\chapter{REFERENCIAS:}\label{referencias}}

\url{https://www.dian.gov.co/ciiu/Documents/Resolucion_000139_21_Nov_2012.pdf}

\url{https://linea.ccb.org.co/descripcionciiu/}

\url{https://siis.ia.supersociedades.gov.co/}

\url{https://www.supersociedades.gov.co/delegatura_aec/Paginas/Base-completa-EF-2019.aspx}

\backmatter
\end{document}
